\part{Baggrund og kontekst}\label{part:baggrund}
\epigraph{\itshape
Danmark mangler allerede nu kvalificeret arbejdskraft, og manglen vil vokse markant de kommende år. Tal fra Arbejderbevægelsens Erhvervsråd viser, at der de kommende ti år vil komme til at mangle omkring 70.000 faglærte. Alene jern- og metalindustriend vil mangle 30.000 faglærte, og tæller vi handels- og kontorområdet med, kan vi lægge yderligere 12.000 oven i. Udover faglærte vil der komme til at mangle flere end 60.000 personer med korte og mellemlange videregående uddannelser.

Samtidig vil omkring 55.000 personer med en lang videregående uddannelse være i overskud i år 2030.
}
{Claus Jensen, forbundsformand, Dansk Metal og Kim Simonsen, forbundsformand, HK, kronik i Berlingske Tidende, 3. april \citeyear{simonsenLadOsGore2016}}

Det er ikke kun politikere, der er interesserede i, hvordan samfundets udvælgesesprocesser og prioriterings\ mekanismer er skruet sammen.
Om man taler om en “overuddannet” befolkning, eller italesætter arbejdsløse akademikere som “dovne”, skorter der ikke med kritik af hvordan man aktuelt har indstillet sorteringsmaskinen.

Nu er dette næppe noget nyt.
Uddannelse har en bred berøringsflade i samfundet, og påkalder sig dermed bred opmærksomhed.
Om det er fordi, man vil gøre de unge til dydige borgere: eller ser uddannelse som en samfundsmæssig udligner; eller vil have flere med på selvudviklingsbølgen \cite{juulDiskurserOmUngdom2013} \todo{dobbelttjek kilden på den opsummering, men det er ca rigtigt. tror jeg}

Diskursen omkring ungdommens uddannelse er således ikke en statisk størrelse, men ændrer sig med tiden og tidens ånd.
Dette vil jeg uddybe i det følgende afsnit, for at kunne kontekstualisere de computerdrevne analyser jeg vil udarbejde.

Jeg starter med en kort gennemgang af nyere forskning, der berører mit emne og forskningsfelt.
Derefter vil jeg drøfte diskursive tendenser i dansk (ungdoms)uddannelsespolitik fra 1950erne til i dag.
For bedre at kunne kvalificere min analyse af erhversuddannelsernes omtale, vil jeg runde af med et overblik over af politiske reformer af de danske erhvervsuddannelser.

\chapter{En gennemgang af forskningslandskabet}\label{sec:litreview}
Jeg vil i det følgende gennemgå nyere forskning indenfor erhvervsuddannelserne.
Jeg vil også kort nævne arbejde indenfor parliamentarisk forskning, samt forskning i hvordan machine learning kan anvendes i et sociologisk perspektiv.

\section{Erhvervsuddannelse}
\todo{int. forskning -  sorting hat}
Jeg holder mig primært til forskning i dansk kontekst i denne oversigt.
Der er lavet nogle komparative analyser af dansk EUD i både skandinavisk og europæisk perspektiv, men dette forekommer mig at falde udenfor specialets fokus.
Der er dog noget arbejde der ser på sorteringsmekanismerne i uddannelsespolitik jeg vil nævne


\citeauthor{juulErhvervsuddannelserneForsomtForskningsomrade2004} beskriver erhvervsuddannelserne som underundersøgte i dansk sammenhæng \autocite{juulErhvervsuddannelserneForsomtForskningsomrade2004}.
Hun har forsøgt at afhjælpe dette, og har forsket i de danske erhvervsuddannelser i omkring 20 år \autocite{IdaJuulPublikationer}.
Som eksempel kan nævnes \citetitle{juulDiskurserOmUngdom2013}, der tager et historisk blik på de diskurser om ungdomsliv der afspejles i policy-tekster om ungdomsuddannelser.
Til forskel fra mit perspektiv, ser \citeauthor{juulDiskurserOmUngdom2013} på teksterne som konsensuspapirer, og foretager en dokumentanalyse af disse \autocite{juulDiskurserOmUngdom2013}.
Hvor \citeauthor{juulDiskurserOmUngdom2013} specifikt ikke ser på hvordan disse konsensuspapirer er opstået, er det netop dette jeg ønsker at undersøge.

Der undersøges en del omkring de unges forhold til ungdomsuddannelserne. CFU, VIVE, EVA og Undervisningsministeriet\todo{kilde: eksempler pw undersøgelser af uud.} foretager jævnligt undersøgelser omkring hvilke ungdomsuddannelser de unge vælger, og hvorfor.
Dette har muligvis en nytteværdi for beslutningstagere, men jeg ser dette som mindre relevant for mine undersøgelser.

Morten Ejrnæs ser lyst på uligheden (eller mangel på samme)

\section{Parliamentariske undersøgelser}\label{sec:review-parl}

En stor del af inspirationen til min forskningsinteresse kommer fra en præsentation ved IMC, Århus Universitet \autocite{interactingmindscentreaarhusuniversityNLPWorkshopIMC2019}. Jan Kostkan og Malte Lau Pedersen præsenterede analyser af ideers nyhedsværdi i en tjekkisk parliamentarisk kontekst\footnote{Det var også her jeg blev bevidst, at folketingstalerne var tilgængelige for analyse}.

Flere undersøgelser i dansk regi der foretager en massebehandling af folketingstaler har dog været svære at finde i store mængder.
\citeauthor{hjorthEstablishmentResponsesPopulist} undersøger hvordan danske politikere forholdte sig til Dansk Folkeparti. Med udgangspunkt i en tekst-som-data tilgang blev folketingstaler fra 1997 til 2004 analyseret ud fra en superviseret machine learning metodologi\footnote{Jeg uddyber disse begreber i mit metodeafsnit}.

Der er til gengæld foretaget en del indholdsanalyser af specifikke emner i dansk regi.

I \citetitle{andersenTaenkIkkePa2012} (\citeyear{andersenTaenkIkkePa2012}) viser \citeautor{andersenTaenkIkkePa2012} hvordan statsminister Løkke Rasmussens udtalelser omkring “ghettoer” er medvirkende til, at skabe et verdens- og selvbillede for beboere i de omtalte områder.
Deres analyse tager udgangspunkt i en nærlæsning af enkelte af statsminsterens taler.

Internationalt er der dog foretaget flere undersøgelser.

\autocite{prokschPositionTakingEuropean2010} ser på hvordan medlemmer i Europaparliamentet forholder sig til \ldots

\autocite{laverExtractingPolicyPositions2003}

\autocite{quinnHowAnalyzePolitical2010} er en metodologisk beskrivelse af forfatternes arbejde ved at udvikle en \textit{topic model} for deres analyse af taler i det amerikanske senatet.

\autocite{slapinScalingModelEstimating2008} beskriver en måde, hvorpå man kan arbejde med tidsserier ved computeranalyser af tekst.

historiske undersøgelser der gør brug af nlp/ml - 19. årh. england, 17? årh. frankrige

\section{Algoritmisk sociologi}\label{sec:review-compsoc}

\cite{evansMachineTranslationMining2016}\todo{uddyb soc <-> tekst mining} beskriver, hvad computerbaserede tekstanalyser kan tilføre sociologien, med et kvalitativt perspektiv. 

 \cite{molinaMachineLearningSociology2019} giver et bredere overblik, over hvad såkaldt “machine learning” kan bidrage til sociologisk forskning.
Der er særligt fokus på den kvantitative sociologi, og hvordan man både kan bruge computeranalyser til at krydsvalidere resultater, og til at udforske data for at udarbejde nye hypoteser.

\chapter{Spændinger og tendenser i ungdomsuddannelse i Danmark}
\todo{David: Herfra og ud er der rigtig meget work in progress og skrivetænkning}

Jeg har nævnt det nogle gange anstrengte forhold mellem arbejdsrettet uddannelse, der har til sigte at give konkrete erhvervsrettede kompetencer, og “uddannelse til (over)uddannelse”.

“ny nordisk skole” fadæsen - den levede ikke ret længe
95 procent af alle ungdomsårgange på ungdomsuddannelser i 90erne, revideret med reform 2000 til, at 97?? skulle gennemføre

formåen → bogligt / praktisk

formål → dannelse / uddannelse

læring → arbejde / skole

lighedstanke  → alle kan opnå hvad de wil
individualisering → alle kan blive lige hvad de vil
Men dog: Den sociale arv er stadigvæk gældende; stor grad af kønssegregation


\chapter{Hvad driver udbredelsen af uddannelse}


globalisering, lighed, uforløste potentialer

\section{Folkeskolen}
I (meget) korte træk, er folkeskolens udvikling fra efterkrigstiden præget af... (enhedsskole, PISA -> konkurrencestat)

Magtforholdet mellem uddannelse og dannelse har i den senere tid skiftet til...

\section{Ungdomsuddannelserne}

hurtigstartsbonus 2009 → afskaffet 2018, med virkning 2020
fremdriftsreform 2013 → hurtiogere færdig med videregående uddannelser
hvordan præsenteres det? output/individualitet/human capital
\section{De videregående uddannelser}
\autocite{schoferWorldwideExpansionHigher2005} viser, hvordan de videregående uddannelser er eksploderet 

Jeg vil dog holde mit fokus på erhvervsuddannelserne.
Med det for eje, vil jeg nu gennemgå de centrale reformer af erhvervsuddannelserne i Danamark.
\chapter{Et oprids af reformer af ungdomsuddannelserne}\todo{eller måske bare erhvervsuddannelserne?}
\todo{KILDE: Juul/Knudsen}
Det bærende element --- og måske endda noget 'typisk dansk' i erhvervsuddannelserne de seneste 20 år, er vekseluddannelsen.
Denne tog form i 19??, hvor eleverne gik over til undervisning i dagskoler.
I 1977 kom EFG til, hvor indgangene til de erhvervsrettede ungdomsuddannelser blev tilrettelagt under et samlet grundforløb.
Hovedtanken i den uddannelsespolitik der blev fremført af den socialdemokratiske regering, var at sikre lighed gennem uddannelse.
Ved at 

Erhvervsuddannelserne har skiftet karakter i løbet af 90erne, en udvikling der er fortsat igennem de sidste 20 år.
I 1991 trådte der en EUD-reform i kraft, der afskaffede mesterlæren, gjorde erhvervsuddannelserne selvejede, indførte taxameterordinger og andre markedsorienterede tiltag.
((erhvervsgymnasiale uddannelser!!))


Dette blev fulgt op af projektet uddannelse til alle.
Fremlagt af daværende undervisningsminister Ole Vig Jensen (R) i november 1993, var det centrale mål, at 95 procent af alle unge skulle have en ungdomsuddannelse.
((fulgt op af lov om de fri ungdomsuddannelser))

Jeg agter ikke at gennemføre en dybdegående indholdsanalyse af disse tekster - det er udenfor denne opgaves remit.
Jeg vil dog påpege centrale tendenser der kan spores i disse reformer og efterfølgende justeringer
