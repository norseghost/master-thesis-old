



hvordan præsenteres det? output/individualitet/human capital
\subsection{Spændinger og tendenser i (ungdoms)uddannelse}

formåen --> bogligt / praktisk

formål → dannelse / uddannelse

lighedstanke --> alle kan opnåe hvad de vil
individualisering → alle kan blive lige hvad de vil (ish)
Men dog: Den sociale arv er stadigvæk gældende; stor grad af kønssegregation


\subsection{Hvad driver udbredelsen af uddannelse}


globalisering, lighed, uforløste potentialer

\subsubsection{Folkeskolen}
I (meget) korte træk, er folkeskolens udvikling fra efterkrigstiden præget af... (enhedsskole, PISA -> konkurrencestat)

Magtforholdet mellem uddannelse og dannelse har i den senere tid skiftet til...

\subsubsection{Ungdomsuddannelserne}

\subsubsection{De videregående uddannelser}
Meyer & Schofer \todo{citeauthor} viser, hvordan de videregående uddannelser er eksploderet 

Jeg vil dog holde mit fokus på erhvervsuddannelserne.
Med det for æje, vil jeg nu gennemgå 
\subsection{Et oprids af reformer af ungdomsuddannelserne}
((KILDE: Juul/Hansen))
Det bærende element --- og måske endda noget 'typisk dansk' i erhvervsuddannelserne de seneste 20 år, er vekseluddannelsen.
Denne tog form i 19??, hvor eleverne gik over til undervisning i dagskoler.
I 1977 kom EFG til, hvor indgangene til de erhvervsrettede ungdomsuddannelser blev tilrettelagt under et samlet grundforløb.
Hovedtanken i den uddannelsespolitik der blev fremført af den socialdemokratiske regering, var at sikre lighed gennem uddannelse.
Ved at 

Erhvervsuddannelserne har skiftet karakter i løbet af 90erne, en udvikling der er fortsat igennem de sidste 20 år.
I 1991 trådte der en EUD-reform i kraft, der afskaffede mesterlæren, gjorde erhvervsuddannelserne selvejede, indførte taxameterordinger og andre markedsorienterede tiltag.
((erhvervsgymnasiale uddannelser!!))


Dette blev fulgt op af projektet uddannelse til alle.
Fremlagt af daværende undervisningsminister Ole Vig Jensen (R) i november 1993, var det centrale mål, at 95 procent af alle unge skulle have en ungdomsuddannelse.
((fulgt op af lov om de fri ungdomsuddannelser))

Jeg agter ikke at gennemføre en dybdegående indholdsanalyse af disse tekster - det er udenfor denne opgaves remit.
Jeg vil dog påpege centrale tendenser der kan spores i disse reformer og efterfølgende justeringer
