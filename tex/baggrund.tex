\section{Baggrund og kontekst}

\begin{quotation}
Danmark mangler allerede nu kvalificeret arbejdskraft, og manglen vil vokse markant de kommende år. Tal fra Arbejderbevægelsens Erhvervsråd viser, at der de kommende ti år vil komme til at mangle omkring 70.000 faglærte. Alene jern- og metalindustrien vil mangle 30.000 faglærte, og tæller vi handels- og kontorområdet med, kan vi lægge yderligere 12.000 oven i. Udover faglærte vil der komme til at mangle flere end 60.000 personer med korte og mellemlange videregående uddannelser.

Samtidig vil omkring 55.000 personer med en lang videregående uddannelse være i overskud i år 2030.
\end{quotation}
--- Claus Jensen, forbundsformand, Dansk Metal og Kim Simonsen, forbundsformand, HK, kronik i Berlingske Tidende, 3. april 2016
\subsection{En gennemgang af forskningslandskabet}

holder mig til forskning i dansk kontekst

Ida Juul \todo{citeauthor} beskriver erhvervsuddannelserne som underundersøgte i dansk sammenhæng.


\subsection{Spændinger og tendenser i (ungdoms)uddannelse}
også her er fokuset danmark
hurtigstartsbonus 2009 → afskaffet 2018, med virkning 2020
fremdriftsreform 2013 → hurtiogere færdig med vidergående uddannelser
hvordan præsenteres det? output/individualitet/human capital

95 \perc af alle ungdomsårgange på ungdomsuddannelser i 90erne, revideret med reform 2000 til, at 97?? skulle gennemføre

formåen --> bogligt / praktisk

formål → dannelse / uddannelse

læring --> arbejde / skole

lighedstanke --> alle kan opnåe hvad de vil
individualisering → alle kan blive lige hvad de vil (ish)
Men dog: Den sociale arv er stadigvæk gældende; stor grad af kønssegregation


\subsection{Hvad driver udbredelsen af uddannelse}


globalisering, lighed, uforløste potentialer

\subsubsection{Folkeskolen}
I (meget) korte træk, er folkeskolens udvikling fra efterkrigstiden præget af... (enhedsskole, PISA -> konkurrencestat)

Magtforholdet mellem uddannelse og dannelse har i den senere tid skiftet til...

\subsubsection{Ungdomsuddannelserne}

\subsubsection{De videregående uddannelser}
Meyer & Schofer \todo{citeauthor} viser, hvordan de videregående uddannelser er eksploderet 

Jeg vil dog holde mit fokus på erhvervsuddannelserne.
Med det for æje, vil jeg nu gennemgå 
\subsection{Et oprids af reformer af ungdomsuddannelserne}
((KILDE: Juul/Hansen))
Det bærende element --- og måske endda noget 'typisk dansk' i erhvervsuddannelserne de seneste 20 år, er vekseluddannelsen.
Denne tog form i 19??, hvor eleverne gik over til undervisning i dagskoler.
I 1977 kom EFG til, hvor indgangene til de erhvervsrettede ungdomsuddannelser blev tilrettelagt under et samlet grundforløb.
Hovedtanken i den uddannelsespolitik der blev fremført af den socialdemokratiske regering, var at sikre lighed gennem uddannelse.
Ved at 

Erhvervsuddannelserne har skiftet karakter i løbet af 90erne, en udvikling der er fortsat igennem de sidste 20 år.
I 1991 trådte der en EUD-reform i kraft, der afskaffede mesterlæren, gjorde erhvervsuddannelserne selvejede, indførte taxameterordinger og andre markedsorienterede tiltag.
((erhvervsgymnasiale uddannelser!!))


Dette blev fulgt op af projektet uddannelse til alle.
Fremlagt af daværende undervisningsminister Ole Vig Jensen (R) i november 1993, var det centrale mål, at 95 procent af alle unge skulle have en ungdomsuddannelse.
((fulgt op af lov om de fri ungdomsuddannelser))

Jeg agter ikke at gennemføre en dybdegående indholdsanalyse af disse tekster - det er udenfor denne opgaves remit.
Jeg vil dog påpege centrale tendenser der kan spores i disse reformer og efterfølgende justeringer
