\part{Baggrund og kontekst}\label{part:baggrund}
\epigraph{\itshape
Danmark mangler allerede nu kvalificeret arbejdskraft, og manglen vil vokse markant de kommende år. Tal fra Arbejderbevægelsens Erhvervsråd viser, at der de kommende ti år vil komme til at mangle omkring 70.000 faglærte. Alene jern- og metalindustriend vil mangle 30.000 faglærte, og tæller vi handels- og kontorområdet med, kan vi lægge yderligere 12.000 oven i. Udover faglærte vil der komme til at mangle flere end 60.000 personer med korte og mellemlange videregående uddannelser.

Samtidig vil omkring 55.000 personer med en lang videregående uddannelse være i overskud i år 2030.
}
{Claus Jensen, forbundsformand, Dansk Metal og Kim Simonsen, forbundsformand, HK, kronik i Berlingske Tidende, 3. april \citeyear{simonsenLadOsGore2016}}

Det er ikke kun politikere, der er interesserede i, hvordan samfundets udvælgelsesprocesser og prioriterings\ mekanismer er skruet sammen.
Om man taler om en “overuddannet” befolkning, eller italesætter arbejdsløse akademikere som “dovne”, skorter der ikke med kritik af hvordan man aktuelt har indstillet sorteringsmaskinen.

Nu er dette næppe noget nyt.
Uddannelse har en bred berøringsflade i samfundet, og påkalder sig dermed bred opmærksomhed.
Om det er fordi, man vil gøre de unge til dydige borgere\todo{kilde: udd som socialisering af borgere}; eller ser uddannelse som en samfundsmæssig udligner\todo{kilde: udd som udlignende faktor}; eller ser uddannelse som vejen til at individet kan opfylde sine samfundsmæssige forpligtelser \cite[s. 9]{juulDiskurserOmUngdom2013}.

Diskursen omkring ungdommens uddannelse er således ikke en statisk størrelse, men ændrer sig med tiden og tidens ånd.
Dette vil jeg uddybe i det følgende afsnit, for at kunne kontekstualisere de computerdrevne analyser jeg vil udarbejde.

Jeg starter med en kort gennemgang af nyere forskning, der berører mit emne og forskningsfelt.
Derefter vil jeg drøfte diskursive tendenser i dansk (ungdoms)uddannelsespolitik fra 1950erne til i dag.
For bedre at kunne kvalificere min analyse af erhversuddannelsernes omtale, vil jeg runde af med et overblik over af politiske reformer af de danske erhvervsuddannelser.

\chapter{En gennemgang af forskningslandskabet}\label{sec:litreview}

Jeg vil i det følgende gennemgå nyere forskning indenfor erhvervsuddannelserne.
Jeg vil også kort nævne arbejde indenfor parliamentarisk forskning, samt forskning i hvordan machine learning kan anvendes i et sociologisk perspektiv.

\section{(Ungdoms)uddannelse og social reproduktion}
Jeg holder mig primært til forskning i dansk kontekst i forhold til erhvervsuddannelserne.
Der er lavet nogle komparative analyser af dansk EUD i både skandinavisk og europæisk perspektiv, men dette forekommer mig at falde udenfor specialets fokus.
Der er dog noget arbejde der ser på sorteringsmekanismerne i uddannelsespolitik jeg vil nævne.
\todo{int. forskning -  sorting hat}

\citeauthor{juulErhvervsuddannelserneForsomtForskningsomrade2004} beskriver erhvervsuddannelserne som underundersøgte i dansk sammenhæng \autocite{juulErhvervsuddannelserneForsomtForskningsomrade2004}.
Hun har forsøgt at afhjælpe dette, og har forsket i de danske erhvervsuddannelser i omkring 20 år \autocite{IdaJuulPublikationer}.
Som eksempel kan nævnes \citetitle{juulDiskurserOmUngdom2013}, der tager et historisk blik på de diskurser om ungdomsliv der afspejles i policy-tekster om ungdomsuddannelser.
Til forskel fra mit perspektiv, ser \citeauthor{juulDiskurserOmUngdom2013} på teksterne som konsensuspapirer, og foretager en dokumentanalyse af disse \autocite{juulDiskurserOmUngdom2013}.
Hvor \citeauthor{juulDiskurserOmUngdom2013} specifikt ikke ser på hvordan disse konsensuspapirer er opstået, er det netop dette jeg ønsker at undersøge.

I temanummeret \citetitle{cangerTemaErhvervsuddannelserMellem2016} af Dansk pædagogisk tidsskrift drøftes reformen af erhvervsuddannelserne i 2014.
Dette sættes i forhold til en generel reformiver i samfundet...\todo{uddyb DpT EUD}

Forskningsprojektet \citetitle{nord-vetFutureVocationalEducation} ser på historiske kontekster for nuværende og fremtidige udfordringer ved erhvervsuddannelser i de nordiske lande.
Projektet har især fokus på erhvervsuddannelsernes tiltagende dobbeltrolle, hvor de samtidigt med at uddanne til arbejdsmarkedet, også er adgangsgivende for højere uddannelse.

Der undersøges en del omkring de unges forhold til ungdomsuddannelserne. CFU, VIVE, EVA og Undervisningsministeriet\todo{kilde: eksempler på undersøgelser af uud.} foretager jævnligt undersøgelser omkring hvilke ungdomsuddannelser de unge vælger, og hvorfor.
Dette har muligvis en nytteværdi for beslutningstagere, men mit fokus er ikke på de unges grundlag for at vælge uddannelser; men politikeres omtale af uddannelserne.
Jeg har i indledningen refereret til en opgørelse lavet af \citeauthor{danmarksstatistikErhvervsuddannelserDanmark20192019}.
Denne udmærker sig ved at give en kort historisk kontekst for fortolkning af tallene \todo{sidetal}(\citeyear[s. ???]{danmarksstatistikErhvervsuddannelserDanmark20192019}).
Der understreges også, at eleverne på ungdomsuddannelserne ikke nødvendigvis er “unge“.
Gennemsnitsalderen for elever på erhvervsuddannelserne er fx ?? \todo{gennemsnitsalder elever erhvervsudd}, hvilket er påfaldende i international sammenhæng.
Til sammenligning er elever på EUD i gennemsnit ??? i \todo{land til sammenligning alder elever EUD}

\todo[color=green!40]{David: Er det følgende relevant? Gør uligheden mere konkret i hvert fald}
Chanceulighed og social reproducktion er der dog set en del på i dansk kontekst.

Morten Ejrnæs ser lyst på uligheden (eller mangel på samme)

David Reimer?

Ida Gran Andersen?

\subsection{Uddannelsessystemet som sorteringsmekanismer}

Sorting hat \todo[color=green!40]{David: Har svært ved at finde noget med dette}

\section{Parliamentariske undersøgelser}\label{sec:review-parl}\todo{litrev: uddyb enkelte bidrag}

En stor del af inspirationen til min forskningsinteresse kommer fra en præsentation ved IMC, Århus Universitet \autocite{interactingmindscentreaarhusuniversityNLPWorkshopIMC2019}. Jan Kostkan og Malte Lau Pedersen præsenterede analyser af ideers nyhedsværdi i en tjekkisk parliamentarisk kontekst\footnote{Det var også her jeg blev bevidst, at folketingstalerne var tilgængelige for analyse}.

Flere undersøgelser i dansk regi der foretager en massebehandling af folketingstaler har dog været svære at finde i store mængder.
Et eksempel er \citeauthor{hjorthEstablishmentResponsesPopulist, der} undersøger hvordan danske politikere forholdte sig til Dansk Folkeparti. Med udgangspunkt i en tekst-som-data tilgang blev folketingstaler fra 1997 til 2004 analyseret ud fra en superviseret machine learning metodologi\footnote{Jeg uddyber disse begreber i mit metodeafsnit}.

Der er til gengæld foretaget en del indholdsanalyser af specifikke emner i dansk regi.

I \citetitle{andersenTaenkIkkePa2012} (\citeyear{andersenTaenkIkkePa2012}) viser \citeauthor{andersenTaenkIkkePa2012} hvordan statsminister Løkke Rasmussens udtalelser omkring “ghettoer” er medvirkende til, at skabe et verdens- og selvbillede for beboere i de omtalte områder.
Deres analyse tager udgangspunkt i en nærlæsning af enkelte af statsminsterens taler.

Internationalt er der dog foretaget flere undersøgelser, hvor flere samtidigt undersøger og udvikler de metodiske redskaber.

I \citetitle{prokschPositionTakingEuropean2010} ser forfatterne på politiske positioner blandt medlemmerne af Europaparliamentet.
Ved hjælp af ordfrekvensanalyser udledes der niveauer af politisk uenighed, der forsøges gjort op på en højre/venstre politisk akse.
Det viser sig dog, at de politiske konflikter i større grad er baseret på EU-samarbejdets karakter \autocite{prokschPositionTakingEuropean2010}.

I \citetitle{laverExtractingPolicyPositions2003} 

\autocite{quinnHowAnalyzePolitical2010} er en metodologisk beskrivelse af forfatternes arbejde ved at udvikle en \textit{topic model} for deres analyse af taler i det amerikanske senatet.

\autocite{slapinScalingModelEstimating2008} beskriver en måde, hvorpå man kan arbejde med tidsserier ved computeranalyser af tekst.

Der er også historiske undersøgelser der gør brug af computeranalyser af tekst.
For eksempel har \citeauthor{barronIndividualsInstitutionsInnovation2018} undersøgt, hvordan det første franske parliament efter den franske revolution forholdt sig til nye ideer.
Gennem analysen viser de, at de store talere fanget folket; men meget af den faktiske indflydelse på lovgivningen faldt til dem, der mestrede komite-arbejdet \autocite{barronIndividualsInstitutionsInnovation2018}. 
 
- 19. årh. england

\section{Algoritmisk sociologi}\label{sec:review-compsoc}

\cite{evansMachineTranslationMining2016} beskriver, hvad computerbaserede tekstanalyser kan tilføre sociologien, med et kvalitativt perspektiv.
De primære fordele de henviser til, er muligheden for at kune bearbejde store mengder data, sammen med 

 \cite{molinaMachineLearningSociology2019} giver et bredere overblik, over hvad såkaldt “machine learning” kan bidrage til sociologisk forskning.
Der er særligt fokus på den kvantitative sociologi, og hvordan man både kan bruge computeranalyser til at krydsvalidere resultater, og til at udforske data for at udarbejde nye hypoteser.

\chapter{Nyere spændinger og tendenser i ungdomsuddannelse i Danmark}

Med et billede af den senere forskning jeg bygger på i dette speciale, vil jeg nu ridse op hvordan vi kom frem til der, hvor vi er i dag.
Da mit fokus for specialet er erhvervsuddannelserne set i forhold til de gymnasiale uddannelser, vil denne udlægning også tage udgangspunkt i erhvervsuddannelserne\todo[color=green!40]{David: Eller bør jeg fordele sol og vind ligeligt mellem EUD/Gym?}

Jeg har nævnt det nogle gange anstrengte forhold mellem arbejdsrettet uddannelse, der har til sigte at give konkrete erhvervsrettede kompetencer, og “uddannelse til (over)uddannelse”.
Dette udspiller sig til en vis grad også indenfor erhvervsuddannelserne selv, der har haft en dobbeltrolle siden 19??.
Erhvervsuddannelserne har, som nævnt ovenfor, i stigende grad skulle give både erhvervsrettede kompetencer og adgang til højere uddannelse,

\citeauthor{bondergaardHistoricalEmergenceKey2014} foreslår følgende grove perioder for dansk erhvervsuddannelse frem til 1990 \autocite[s. 7f]{bondergaardHistoricalEmergenceKey2014}

← 1945: \todo[color=green!40]{David: Hvor meget kontekst for perioderne jeg vil undersøge?} lavenes mesterlære, fra monopol til deregulering til re-regulering.
Denne periode kommer forud for mit fokusområde, men det er allerede her, at grundlaget for den erhvervsuddannelse vi ser i dag bliver lagt, med et bredt samarbejde mellem argbejdstagere, arbejdsgivere og staten.
Samtidig ser teksniske skoler en stor vækst, men gymnasierne er forbeholdt de mere priviligerede unge \autocite[s. 9]{bondergaardHistoricalEmergenceKey2014}.
Det er også her, en formel skole-undervisning i stigende grad bliver en del af uddannelsen \autocite[s. 15ff]{bondergaardHistoricalEmergenceKey2014}. 

Ved opløsning af lavene i 1862 ændrede forholdene sig.
Der var nu ingen regulering af hvem der kunne tage lærlinge; hvilket medførte et skift incitamenter for  “mestrene”, der kom til at behandle lærlinge som billig arbejdskraft, på bekostning af deres uddannelse.\todo{omskrives}

1945 — 1967: udvidelse og specialisering af mesterlæren



1967 — 1990: mesterlæren i konflikt med den nye erhvervsuddannelse


I årene efter anden verdenskrig var de danske ungdomsuddannelser præget af...
Sortering til akademisk spor allerede ved... En sortering, der kraftigt favoriserede middelklassens og borgerskabets børn.

De erhvervsrettede uddannelser har rødder i de gamle lav, baserede sig på mesterlæren 
lavenes monopolpå uddannelse ophævet, efter nogle år med ingen egulering delvis statslig kontrol fra det sene  1800tal

1956: vekseluddannelsen indføres

1976: EFG indføres, som et borgerligt svar på de kritikker der var rejst af venstrefl

\citeauthor{juulDiskurserOmUngdom2013} beskriver i \citetitle{juulDiskurserOmUngdom2013} hvordan denne (ubevidste?) favorisering af middelklassens dyder og verdier blev
til en form for skabelon for ungdomslivet.
De unge skulle i skole, både for at lære noget nyttigt, men også for at få dem ud af cafeerne og bodegaerne.


I løbet af 60(?)erne var der en gryende optimisme omkring uddannelse som en stor udligner\todo{hos hvem? hvornår}, der ville hæve den gemene hob op; oplysningstidens idealforståelser om den oplyste samfundsborger og dennes rolle i samfundet fik en parallel, om ikke en renaissance....

80erne så en ny tilgang til uddannelse.
I den neoliberale konkurrencestat skulle uddannelse tjene sanfundet direkte...\todo{siger hvem?}


“ny nordisk skole” fadæsen - den levede ikke ret længe
95 procent af alle ungdomsårgange på ungdomsuddannelser i 90erne, revideret med reform 2000 til, at 97?? skulle gennemføre

formåen → bogligt / praktisk

formål → dannelse / uddannelse

læring → arbejde / skole

lighedstanke  → alle kan opnå hvad de wil
individualisering → alle kan blive lige hvad de vil
Men dog: Den sociale arv er stadigvæk gældende; stor grad af kønssegregation



globalisering, lighed, uforløste potentialer

\section{Ungdomsuddannelserne}\todo{skal skrives ind i det foregående - gamle noter}

hurtigstartsbonus 2009 → afskaffet 2018, med virkning 2020
fremdriftsreform 2013 → hurtiogere færdig med videregående uddannelser
hvordan præsenteres det? output/individualitet/human capital
\section{De videregående uddannelser}
\autocite{schoferWorldwideExpansionHigher2005} viser, hvordan de videregående uddannelser er eksploderet 

Jeg vil dog holde mit fokus på erhvervsuddannelserne.
Med det for eje, vil jeg nu gennemgå de centrale reformer af erhvervsuddannelserne i Danamark.
\section{Et oprids af reformer af ungdomsuddannelserne}\todo{eller måske bare erhvervsuddannelserne?}
\todo{KILDE: Juul/Knudsen}
Det bærende element --- og måske endda noget 'typisk dansk' i erhvervsuddannelserne de seneste 20 år, er vekseluddannelsen.
Denne tog form i 19??, hvor eleverne gik over til undervisning i dagskoler.
I 1977 kom EFG til, hvor indgangene til de erhvervsrettede ungdomsuddannelser blev tilrettelagt under et samlet grundforløb.
Hovedtanken i den uddannelsespolitik der blev fremført af den socialdemokratiske regering, var at sikre lighed gennem uddannelse.
Ved at 

Erhvervsuddannelserne har skiftet karakter i løbet af 90erne, en udvikling der er fortsat igennem de sidste 20 år.
I 1991 trådte der en EUD-reform i kraft, der afskaffede mesterlæren, gjorde erhvervsuddannelserne selvejede, indførte taxameterordinger og andre markedsorienterede tiltag.
((erhvervsgymnasiale uddannelser!!))


Dette blev fulgt op af projektet uddannelse til alle.
Fremlagt af daværende undervisningsminister Ole Vig Jensen (R) i november 1993, var det centrale mål, at 95 procent af alle unge skulle have en ungdomsuddannelse.
((fulgt op af lov om de fri ungdomsuddannelser))

Jeg agter ikke at gennemføre en dybdegående indholdsanalyse af disse tekster - det er udenfor denne opgaves remit.
Jeg vil dog påpege centrale tendenser der kan spores i disse reformer og efterfølgende justeringer
