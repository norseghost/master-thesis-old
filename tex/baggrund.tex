\renewcommand*{\afterpartskip}{
\vfil\epigraph{\itshape
Danmark mangler allerede nu kvalificeret arbejdskraft, og manglen vil vokse markant de kommende år. Tal fra Arbejderbevægelsens Erhvervsråd viser, at der de kommende ti år vil komme til at mangle omkring 70.000 faglærte. Alene jern- og metalindustriend vil mangle 30.000 faglærte, og tæller vi handels- og kontorområdet med, kan vi lægge yderligere 12.000 oven i. Udover faglærte vil der komme til at mangle flere end 60.000 personer med korte og mellemlange videregående uddannelser.

Samtidig vil omkring 55.000 personer med en lang videregående uddannelse være i overskud i år 2030.
}
{Claus Jensen, forbundsformand, Dansk Metal og Kim Simonsen, forbundsformand, HK, kronik i Berlingske Tidende, 3. april \citeyear{simonsenLadOsGore2016}}
}

\part{Baggrund og kontekst}\label{part:baggrund}

\chapter{Markedskræfter og medborgerskab}\label{chap:marked}

Det er ikke kun politikere, der blander sig i, hvordan samfundets udvælgelsesprocesser og prioriterings\ mekanismer er skruet sammen.
Om man taler om en “overuddannet” befolkning, eller italesætter arbejdsløse akademikere som “dovne”, skorter der ikke med kritik af hvordan man aktuelt har indstillet sorteringsmaskinen.

Nu er dette næppe noget banebrydende.
Uddannelse har en bred berøringsflade i samfundet, og påkalder sig dermed bred opmærksomhed.
Og uddannelse af “ungdommen” — disse voksne under tilblivelse — især.
Diskursen omkring ungdommens uddannelse er heller ikke en statisk størrelse, men ændrer sig med tiden og tidens ånd.
I efterkrigstiden var fokus fx, at sikre de unge fra vilkårlighed ved at informere og vejlede \autocite[s 17ff]{juulDiskurserOmUngdom2013}.
I 70erne ses uddannelse som en samfundsmæssig udligner, der ikke blot skal tjene erhvervslivet, men også hæve befolkningens kvalifikationer ud over det boglige \autocite[s 15ff]{juulDiskurserOmUngdom2013}.
Og fra 90-tallet ser vi et fokus på uddannelse som vejen til at individet kan opfylde sin samfundsmæssige forpligtelser på arbejdsmarkedet \autocite[s. 18]{juulDiskurserOmUngdom2013}.

Disse skiftende holdninger til uddannelse vil jeg uddybe i denne del af specialet, for at kunne kontekstualisere de computerdrevne analyser jeg vil udarbejde.

Jeg starter med en kort gennemgang af nyere forskning, der berører mit emne og forskningsfelt.
Derefter vil jeg drøfte diskursive tendenser i dansk (ungdoms)uddannelsespolitik fra 1970erne til i dag, med særligt fokus på erhvervsuddannelserne.

\chapter{En gennemgang af forskningslandskabet}\label{sec:litreview}

Jeg vil i det følgende gennemgå nyere forskning indenfor erhvervsuddannelserne specifikt, men også ungdomsuddannelser generelt.
Jeg vil derefter kort nævne arbejde indenfor parliamentarisk forskning.
Kapitlet afsluttes med udlægninger af, hvordan tekstmining kan anvendes i et sociologisk perspektiv.

\section{(Ungdoms)uddannelse og social reproduktion}
I det følgende præsenteres noget arbejde omkring erhvervs- og ungdomsuddannelserne.
Jeg holder mig primært til forskning i dansk kontekst i forhold til erhvervsuddannelserne.
Der er lavet nogle komparative analyser af dansk EUD i både skandinavisk og europæisk perspektiv, men dette forekommer mig at falde udenfor specialets fokus.
Der er dog noget arbejde der ser på sorteringsmekanismerne i uddannelsespolitik
jeg vil nævne.

\citeauthor{juulErhvervsuddannelserneForsomtForskningsomrade2004} beskriver erhvervsuddannelserne som underundersøgte i dansk sammenhæng \autocite{juulErhvervsuddannelserneForsomtForskningsomrade2004}.
Hun har forsøgt at afhjælpe dette, og har forsket i de danske erhvervsuddannelser i omkring 20 år \autocite{IdaJuulPublikationer}.
Som eksempel kan nævnes \citetitle{juulDiskurserOmUngdom2013}, der tager et historisk blik på de diskurser om ungdomsliv der afspejles i policy-tekster om ungdomsuddannelser.
Til forskel fra mit perspektiv, ser \citeauthor{juulDiskurserOmUngdom2013} på teksterne som konsensuspapirer, og foretager en dokumentanalyse af disse \autocite{juulDiskurserOmUngdom2013}.
Hvor \citeauthor{juulDiskurserOmUngdom2013} specifikt ikke ser på hvordan denne type konsensuspapirer opstår, er det netop dette jeg ønsker at undersøge.

I temanummeret \citetitle{cangerTemaErhvervsuddannelserMellem2016} af Dansk pædagogisk tidsskrift drøftes virkningerne af reformen af erhvervsuddannelserne i 2014.
Denne sættes i forhold til en generel reformiver omkring uddannelser i dag, hvor også tidsintervallerne mellem reformerne forkortes \autocite[s. 2]{cangerTemaErhvervsuddannelserMellem2016}. 

Jeg hæfter mig især ved artiklen \citetitle{jorgensenReformenAfErhvervsuddannelserne2016}, hvor \citeauthor{jorgensenReformenAfErhvervsuddannelserne2016} påpeger, at reformen af 2014 i høj grad handler om at imødekomme utilsigtede konsekvenser af tidligere reformer.
De utilsigtede konsekvenser omkring erhvervsuddannelserne er mange, da disse i senere tid har haft berøringsflader til sociale såvel som arbejdsmæssige politiske arenaer.

Forskningsprojektet \citetitle{nord-vetFutureVocationalEducation} ser på historiske kontekster for nuværende og fremtidige udfordringer ved erhvervsuddannelser i de nordiske lande.
Projektet har især fokus på erhvervsuddannelsernes tiltagende dobbeltrolle, hvor de samtidigt med at uddanne til arbejdsmarkedet, også er adgangsgivende for højere uddannelse.

Der undersøges en del omkring de unges forhold til ungdomsuddannelserne.
CFU, VIVE, EVA og Undervisningsministeriet foretager jævnligt undersøgelser omkring hvilke ungdomsuddannelser de unge vælger, og hvorfor \autocite[se fx.][]{undervisningsministerietOg10Klasseelevernes2017, borne-ogundervisningsministerietHvemOgHvor, danmarksstatistikErhvervsuddannelserDanmark20192019}.
Dette har muligvis en nytteværdi for beslutningstagere, men mit fokus er ikke på de unges grundlag for at vælge uddannelser; men politikeres omtale af uddannelserne.

Jeg har i indledningen refereret til en opgørelse lavet af \citeauthor{danmarksstatistikErhvervsuddannelserDanmark20192019}.
Denne udmærker sig ved at give en kort historisk kontekst for fortolkning af tallene (\citeyear[s. 8ff]{danmarksstatistikErhvervsuddannelserDanmark20192019}).
Der understreges også, at eleverne på erhvervsuddannelserne ikke nødvendigvis er “unge“.
Gennemsnitsalderen for elever på erhvervsuddannelserne er knap 24 år, med variationer indenfor de forskellige uddannelser \autocite[s. 14]{danmarksstatistikErhvervsuddannelserDanmark20192019}.

Chanceulighed og social reproduktion er der set en del på i dansk kontekst.

Jeg henviste i specialets første del til \citeauthor{munkSocialUlighedOg2014}, der beskriver Danmark som et land med overordnet positiv udvikling; men at forældres baggrund stadigvæk er indikerende for, hvilke uddannelser man tager \autocite{munkSocialUlighedOg2014}.
Dette underbygges af \citeauthor{felsbirkelundStructureCausesConsequencesInprogress}, der ser på tracking i en dansk kontekst.
Ved at gennemgå danske registerdata, viser de at de internationale erfaringer i også er gældende i Danmark.
Er ens forældre højtuddannede, er det meget større sandsynlighed for, at blive universitetsstuderende, og man har også højere social status og disponibel indkomst over tid \autocite{felsbirkelundStructureCausesConsequencesInprogress}.

I rapporten \citetitle{thomsenUddannelsesmobilitetDanmark2016}, udgivet af SFI, viser det sig også, at på trods en generel stigning i uddannelsesmobilitet, er der stadig store forskelle i hvem der tager hvilke uddannelser.

For en drøftelse af sorteringsmekanismer til uddannelse, har jeg tidligere refereret til \citetitle{felsbirkelundStructureCausesConsequencesInprogress}; hvor \citeauthor{felsbirkelundStructureCausesConsequencesInprogress} forsøger at afdække korrelationer mellem sociale klasseforskelle og uddannelsesudfald i en dansk kontekst.
Det er et særligt fokus på de langsigtede konsekvenser af hvilket uddannelsesspor man følger, med udgangspunkt i årgangen født i 1975.
De finder, at der er klart flertal af studerende med akademisk uddannede forældre der selv gennemfører en akademisk uddannelse.
Videre viser det sig, at børn hvis forældre har en ungdomsuddannelse, i markant større grad gennemfører en videregående uddannelse end børn der kun har forældre med obligatorisk skolegang.
Dog er der omtrentlig lige stor andel af unge fra begge disse grupperinger der gennemfører en ungdomsuddannelse \autocite[s 12]{felsbirkelundStructureCausesConsequencesInprogress}.
Den langsigtede analyse tyder også på, at opnåelse af en erhvervsuddannelse i høj grad er medvirkende til bedre arbejdsmarkeds- og uddannelsesudfald \autocite[s. 24]{felsbirkelundStructureCausesConsequencesInprogress}.

Det tydelige sammenfald mellem socioøkonomisk baggrund og sortering til specifikke uddannelsesspor er underbygget i en tysk kontekst af \citeauthor{schnepfSortingHatThat2002}, der ser samme variation i optagelse ved tyske uddannelsesspor; med stor ulighed i uddannelsesudfald til følge; forstærket af den tidlige fordeling af uddannelsesspor i Tyskland (\citeyear{schnepfSortingHatThat2002}).
 
\section{Parliamentariske undersøgelser}\label{sec:review-parl}

En stor del af inspirationen til min forskningsinteresse kommer fra en præsentation ved IMC, Århus Universitet \autocite{interactingmindscentreaarhusuniversityNLPWorkshopIMC2019}. Jan Kostkan og Malte Lau Pedersen præsenterede analyser af ideers nyhedsværdi i en tjekkisk parliamentarisk kontekst\footnote{Det var også her jeg blev bevidst, at folketingstalerne var tilgængelige for analyse}.

Flere undersøgelser i dansk regi der foretager en massebehandling af folketingstaler har dog været svære at finde.
Et eksempel er \citeauthor{hjorthEstablishmentResponsesPopulist2018}, der i \citetitle{hjorthEstablishmentResponsesPopulist2018} undersøger hvordan danske politikere forholdt sig til, at Dansk Folkeparti blev en del af det politiske landskab.
Med udgangspunkt i en tekst-som-data tilgang blev folketingstaler fra 1997 til 2004 analyseret ud fra en superviseret machine learning metodologi\footnote{Jeg uddyber de tekniske begreber i mit metodeafsnit} \autocite{hjorthEstablishmentResponsesPopulist2018}.

Der er til gengæld foretaget en del indholdsanalyser af specifikke emner i dansk regi.

I \citetitle{andersenTaenkIkkePa2012} (\citeyear{andersenTaenkIkkePa2012}) viser \citeauthor{andersenTaenkIkkePa2012} hvordan statsminister Løkke Rasmussens udtalelser omkring “ghettoer” er medvirkende til, at skabe et verdens- og selvbillede for beboere i de omtalte områder.
Deres analyse tager udgangspunkt i en nærlæsning af enkelte af statsminsterens taler.

Internationalt er der dog foretaget flere undersøgelser, hvor flere samtidigt undersøger og udvikler de metodiske redskaber.

I \citetitle{prokschPositionTakingEuropean2010} ser forfatterne på politiske positioner blandt medlemmerne af Europaparliamentet.
Ved hjælp af ordfrekvensanalyser udledes der niveauer af politisk uenighed, der forsøges gjort op på en højre/venstre politisk akse.
Det viser sig dog, at de politiske konflikter i større grad er baseret på EU-samarbejdets karakter end over tilhørighed til bestemte politiske fløje \autocite{prokschPositionTakingEuropean2010}.

I \citetitle{laverExtractingPolicyPositions2003} undersøger \citeauthor{laverExtractingPolicyPositions2003} muligheder for at benytte tekst-som-data som en tilgang til at udlede politiske positioner.
De ser endvidere på mulighederne i, at foretage computerbaserede kodninger af tekstmaterialet kontra kodning via mennesketimer, for dermed at gøre fremtidigt analysearbejde mindre omstændigt og tidkrævende \autocite{laverExtractingPolicyPositions2003}.

\citetitle{quinnHowAnalyzePolitical2010} er en metodologisk beskrivelse af forfatternes arbejde ved at udvikle en \textit{topic model} for deres analyse af taler i det amerikanske senatet.
Deres tilgang udleder både de ord, der er indikerende for specifikke emner; samt antal af emner i en samling af tekst \autocite{quinnHowAnalyzePolitical2010}.

\citeauthor{slapinScalingModelEstimating2008} beskriver en måde, hvorpå man kan arbejde med (parti)politiske positioner ved computeranalyser af tekst.
De ser på partiprogrammer fra Tyskland i perioden 1990-2005, og kortlægger partiernes politiske positioner over tid.
Deres tilgang adskiller sig fra \citeauthor{laverExtractingPolicyPositions2003} idet, det ikke er nødvendigt at have et set veldefinerede tekster at beregne referenceværdier ud fra \autocite{slapinScalingModelEstimating2008}.

Der er også historiske undersøgelser der gør brug af computeranalyser af tekst.
For eksempel har \citeauthor{barronIndividualsInstitutionsInnovation2018} undersøgt, hvordan det første franske parliament efter den franske revolution forholdt sig til nye ideer.
Gennem analysen viser de, at de store talere fanget folket; men meget af den faktiske indflydelse på lovgivningen faldt til dem, der mestrede komite-arbejdet i de politiske baglokaler \autocite{barronIndividualsInstitutionsInnovation2018}. 

\section{Algoritmisk sociologi}\label{sec:review-compsoc}
De ovenstående eksempler drager i høj grad på computerbaserede undersøgelsesmetoder; men der findes også arbejde, der ser på den algoritmiske sociologi mere generelt.

\cite{evansMachineTranslationMining2016} beskriver, hvad computerbaserede tekstanalyser kan tilføre sociologien, med et kvalitativt perspektiv.
De ser på både superviserede og ikke-superviserede tilgange til computerbaseret tekstanalyse, i tre sociale dimensioner der kan undersøges ud fra tekstmaterialer.

 \cite{molinaMachineLearningSociology2019} giver et bredere overblik, over hvad såkaldt “machine learning” kan bidrage til sociologisk forskning.
Der er særligt fokus på den kvantitative sociologi, og hvordan man både kan bruge computeranalyser til at krydsvalidere resultater, og til at udforske data for at udarbejde nye hypoteser.

\chapter{Spændinger og tendenser i erhvervsuddannelserne i Danmark}
\todo{for meget fokus på erhvervsuddannelserne??}
Med et billede af den senere forskning jeg bygger på i dette speciale, vil jeg nu ridse op hvordan vi kom frem til der, hvor vi er i dag.
Da mit fokus for specialet er erhvervsuddannelserne set i forhold til de gymnasiale uddannelser, vil denne udlægning også tage udgangspunkt i erhvervsuddannelserne.
Derefter vil jeg beskrive nogle af interne spændinger der har præget — og stadig præger — erhvervsuddannelserne.

\section{Et historisk oprids}

Det følgende vil tage afsæt i \cite{bondergaardHistoricalEmergenceKey2014} og \cite{juulDiskurserOmUngdom2013} for en grov inddeling af perioder indenfor dansk erhvervsuddannelse.

Jeg starter med en (meget) overordnet gennemgang af tendenserne op til hvor min analyse starter; for derefter at gå mere i dybden med mine analyseperioder.

\subsection{Historien før min analyse}

 Grundlaget for den erhvervsuddannelse vi ser i dag bliver lagt allerede før 1945, med et bredt samarbejde mellem arbejdstagere, arbejdsgivere og staten.
De tekniske skoler en stor vækst, men gymnasierne er forbeholdt de mere priviligerede unge \autocite[s. 9]{bondergaardHistoricalEmergenceKey2014}.
Det er også her, en formel skole-undervisning i stigende grad bliver en del af erhvervsuddannelsen \autocite[s. 15ff]{bondergaardHistoricalEmergenceKey2014}. 

Med store ungdomsårgange efter 2. verdenskrig, samt en stor vækst indenfor industriel produktion, er der nu en delvis svækkelse af tidligere restriktioner for antal lærlinge.
Det er også nu vi ser dagskoler blive en integreret og obligatorisk del af undervisningen, der også bliver mere specialiseret.
Undervisningen ivaretages også i højere grad af centralstyrede statslige skoler \autocite[s. 35]{bondergaardHistoricalEmergenceKey2014}.

Nu er “ungdom” også et fænomen der i mindre grad forbeholdes de mere velhavende samfundslag.
Dermed ses det også som en vigtig politisk opgave at vejlede de blivende borgere i, at foretage fornuftige — forstået som i tråd med middelklassens værdier — valg, både kort- og langsigtet \autocite[s. 13ff]{juulDiskurserOmUngdom2013}.

\subsection{1978 — 1990: Uddannelse i fokus}
Grundlaget for hvad der bliver til den erhvervsfaglige grunduddannelse (EFG), lægges allerede fra sidst i 60'erne.
Den implementeres i den anden halvdel af 70erne.
Udarbejdet som en afløser til mesterlæren, hvor indgangene til de erhvervsrettede ungdomsuddannelser tilrettelægges under et samlet grundforløb, ender EFG dog med at køre sideløbende med den. 
Dette afspejler en grundlæggende konflikt omkring udformningen af erhvervsuddannelserne, mellem en mere skole-baseret og tilnærmet gymnasierne; og en mere praksis-orienteret tilgang \autocite[s. 49ff, 57]{bondergaardHistoricalEmergenceKey2014}.

I 1978, hvor jeg starter mine analyser, udgives \citetitle{undervisningsministeriet90SamletUddannelsesplanlaegning1978}, der ser for sig et samfund hvor uddannelse får en voksende betydning \autocite{undervisningsministeriet90SamletUddannelsesplanlaegning1978}.
Denne rapport kommer også med kritik overfor erhvervslivets store indflydelse på uddannelserne.
Der bør være et mere helhedlig fokus, understreges der; med mere almene kvalifikationer fra uddannelse — og et generelt højere uddannelsesniveau i samfundet \autocite[s 18f]{juulDiskurserOmUngdom2013}.
Hovedtanken i den uddannelsespolitik der blev fremført af den socialdemokratiske regering, var at sikre lighed gennem uddannelse.

\subsection{1990 — 2001: Målstyring og selvstyre, under markedsvilkår}

Erhvervsuddannelserne skifter karakter i løbet af 90erne.
I 1990 udgives \citetitle{undervisningsministerietU91DetNye}, der er et kraftigt opgør med tankerne fra U90.
I 1991 trådte der en EUD-reform i kraft, der afskaffede mesterlæren, gjorde erhvervsuddannelserne selvejede, indførte taxameterordinger og andre markedsorienterede tiltag.
Dette for at komme væk fra central detailstyring, der forhindrer nyskabelse og innovation.
Uddannelsens primære fokus er også, at føre de unge hen til arbejdsmarkedet \autocite[s. 19]{juulDiskurserOmUngdom2013}.

\subsection{2001-2014: Ansvar for egen læring, med individet i fokus}

Nu er uddannelse vores forsvar mod globaliseringens trusler.
Det påhviler den enkelte, at tage ansvar for at blive ved med at dygtiggøre sig selv; og hvis man ikke har den påkrævede konkurrenceorientering, skal man til at finde den.
Incitamenter til at skifte disse mentale gear kommer via støtteordninger — fx vejledning — og økonomiske sanktioner

\subsection{2014-2020: Reformernes tidsalder}

Som redaktørene skriver i indledningen til \citetitle{cangerTemaErhvervsuddannelserMellem2016}, er de seneste år præget af en voldsom reformiver omkring uddannelse.
Ofte kommer nye reformer inden den forrige reform får tid til at bundfalde sig
De hyppige reformer fremstilles dels som et udtryk for, at en uddannet befolkning er en af velfærdsstatens forudsætninger; og dels i lyset af, at (ungdoms)uddannelse er blevet en central velfærdspolitisk kamparena.
Som \citeauthor{jorgensenReformenAfErhvervsuddannelserne2016} antyder, har reformerne også aspekter af at være gammel vin på nye flasker.
“Nye” tiltag sættes ind for at forsøge at rette op på den foregående reforms utilsigtede konsekvenser — inden næste reform svinger pendulet tilbage igen \autocite[s.9 ]{jorgensenReformenAfErhvervsuddannelserne2016}.

\section{Spændinger og modsatrettede tendenser}

Jeg indledte denne del af specialet med et udtryk for (det til tider anstrengte) forhold mellem arbejdsrettet uddannelse, der har til sigte at give konkrete erhvervsrettede kompetencer der fører til arbejde; og en mere boglig, skolastisk og generel uddannelse, der åbner for mere (over)uddannelse.

Denne spænding udspiller sig til en vis grad også indenfor erhvervsuddannelserne selv, der længe har haft en dobbeltrolle.
Erhvervs\~uddannelserne har i stigende grad skulle give både erhvervsrettede kompetencer og adgang til højere uddannelse; hvilket tog til i efterkrigstiden \autocite[s. 47ff]{bondergaardHistoricalEmergenceKey2014} og blev stadfæstet ved indførelse af EUX i 2010.


Samtidigt har 95\%-målsætningen; hvor mindst 95\% af en ungdomsårgang skulle gennemføre en ungdomsuddannelse, ført til en ændret demografi hos erhvervsuddannelserne.
Dette er en central del af tankegangen bag “uddannelse for alle”, hvor erhvervsuddannelserne har skulle imødekomme et bredere udsnit af de unge, ved hjælp af øget differentiering \autocite[s. 365f]{aarkrogRummelighedOgSammenhaeng2003}.
\citeauthor{aarkrogRummelighedOgSammenhaeng2003} beskriver videre, hvordan denne differentiering i Erhvervsuddannelsesreform 2000, ved at tage udgangspunkt i den enkelte elev, nu giver eleven en  stor del af, ja ansvar for egen læring, Undervisningen er endvidere forholdsvis abstrakt, og stiller store krav til refleksion \citeyear[s. 367f]{aarkrogRummelighedOgSammenhaeng2003}.
De dominerende pædagogiske diskurser overføres til erhvervsuddannelserne, uden at der skeles til, om de er bedst tjent med en mere direkte, anvendelsesorienteret tilgang\todo{find sidetal} \autocite[s. ]{aarkrogRummelighedOgSammenhaeng2003}.

Derudover har den aktive socialpolitk, der i høj grad betinger sociale ydelser på arbejds- eller uddannelsesvillighed, medført at mange af erhvervsuddannnelsens elever fra omkring årtusindeskiftet har været mere ressourcesvage end tidligere.
Dette har, jævnfør de førnævnte utilsigtede konsekvenser, medført et frafald fra ungdomsuddannelserne der følger kravene fra aktiveringspolitikken \autocite[s.13]{jorgensenReformenAfErhvervsuddannelserne2016}.

Dette er også et udtryk for spændingen mellem erhvervsuddannelsernes karakterisation som “ungdomsuddannelse” og deres funktion i efteruddannelsen af voksne.
Gennemsnitsalderen for elever på erhvervsuddannelserne var knap 24 år i 2018; med uddannelser indenfor området for omsorg, sundhed og pædagogik med den højeste gennemsnitsalder; på knap 30 år \autocite[s. 14]{danmarksstatistikErhvervsuddannelserDanmark20192019}.
Dermed vil et ensidigt fokus på at rekruttere unge kunne gøre erhvervsuddannelserne mindre attraktive for voksne der søger opkvalificering, som beskrevet af \citeauthor{jorgensenReformenAfErhvervsuddannelserne2016} (\citeyear[s. 13]{jorgensenReformenAfErhvervsuddannelserne2016}).

Der er heller ikke samsvar mellem antallet udbudte praktikpladser i forhold til antallet optagede elever; ligesom der ikke er opfyldt de politiske mål for optagelse til erhvervsuddannelserne \autocite[s. 10]{danmarksstatistikErhvervsuddannelserDanmark20192019}.


