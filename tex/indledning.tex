\section{Indledning}
\begin{epigraphs}
\qitem{
De unge stiller i dag store krav og forventninger om udviklingsmuligheder.
Det er samfundets opgave at reagere imødekommende på disse behov.
Hver enkelt af de unge skal have
reelle muligheder for at opnå de bedst mulige uddannelsesmæssige forudsætninger for et rigt voksenliv.

Det er i hele samfundets interesse at gøre en forstærket indsats.
At efterlade en stor del af en ungdomsgeneration uden de nødvendige kvalifikationer er at lade dem i stikken i en fremtid, der stiller stadig større krav til uddannelse.
En stor gruppe unge uden uddannelse er måske frem for alt et problem for demokratiet.
}
{Undervisningsminister Ole Vig Jensen, Redegørelse til folketinget om uddannelse til alle, 11. november 1993}

\qitem{
Vi skal have uddannelser i verdensklasse. Målet er, at eleverne i folkeskolen bliver blandt verdens bedste til læsning, matematik og naturfag. At mindst 95 pct. af de unge får en ungdomsuddannelse i 2015. Og at mindst 50 pct. af en ungdomsårgang får en videregående uddannelse i 2015.
}
{Statsminister \Citeauthor{rasmussenStatsministerAndersFogh2005}, tale ved folketingets åbning, 24. februar 2005}
\end{epigraphs}


Ingen politiker kan undlade at have en holdning til uddannelse.
Fra børnehaven, der er blevet mere eksplicit skoleforberedende, hele vejen til universitetet, der skal imødekomme en verden i global konkurrence.\todo{citation needed}
Jeg akter at gå i dybden omkring ungdomsuddannelserne, hvor der indenfor de seneste 25 år har været 7\todo{mindst!} reformer af forskellig grad.\todo{tjek tal, find kilde }

Ungdomsuddannelserne har mange opgaver\todo{kilder!} at løse i dag.
De skal på samme tid
\begin{itemize}
  \item
    sikre, at Danmark har den kvalificerede arbejdskraft der er behov for, nu og i fremtiden
  \item
    sikre, at Danmark kan begå sig i den internationale konkurrence
  \item
    sikre social mobilitet, således at alle der ønsker en ungdoms- eller videregående uddannelse har mulighed for at få en
  \item
    sikre den danske velfærdsstats overlevelse, ud fra devicen, at uddannelse vil give bedre mulighed for arbejde; og dermed selvforsørgelse \todo{omskrives?}
  \item
    sikre sammenhængskraften i det danske samfund, hvor man vil undgå, at der er grupperinger der er “udenfor” fællesskabet\footnote{Mangel på uddannelse er her set som en proxy-variabel for forskellige grader af “social udsathed” — levealder, overførselsindkomst, livstilssygdomme mm.}\todo{citation needed} - den såkaldte restgruppe\todo{kilde}
  \item
    bidrage til den (fortsatte) dannelse af de unge, til oplyste og refleksive nations- og verdensborgere\todo{citation needed}
\end{itemize}

Flere af de ovenstående formål griber ind i hinanden, og bringer stikord som “socialpolitik”, “human capital' verdensborgere og “globalisering” frem.\todo{that is ikke a good vending}
Undervisningsminister Ole Vig Jensen funderer endda over, om de u(ud)dannede unge kan udgøre en trudsel mod demokratiet, og vel et årti senere lover statsminister Anders Fogh Rasmussen uddannelse 'i verdensklasse'

Der er dog nogle mere skjulte mekanismer på spil uddannelsessystemet.
Uddannelsessystemet er de facto en sorteringsmekanisme - det er det til en vis grad nødt til at være, jævnfør de to første elementer i listen ovenfor.
Men hvad angår den sociale mobilitet, er der også elementer af, en legitimering af strukturelle forskelle \todo{citation needed}.
Dette forstærkes (muligvis) af en form for privilegieblindhed i Danmark.  \todo{skal uddybes, og underbygges}
Der er en udbredt forståelse af Danmark som et forholdsvist fladt samfund, hvor “alle” har mulighed for at blive ”alt”, hvilket kan gøre, at de dokumenterede forskelle i uddannelsesudfald ikke tages alvorligt.\todo{shots fired. need cover}

Derudover er der en faldende tilslutning til erhvervsuddannelserne. Sammen med et højt frafald, er dette bekymrende for flere interessenter.
Dansk Erhverv m fl bekymrer sig fx om, hvorvidt EUD fravælges, blot fordi der går lavstatus i EUD / gymnasiet ses som 'den rigtige' uddannelse.

Og man kan også indstille sorteringsmaskinen 'forkert', hvilket illustreres af det indledende citat nr 2, af formændende for Dansk Metal og HK.
De vil 'gøre op med overuddannelsen', der (i deres øjne) er til hinder for vækst og fremgang i Danmark. 

Jeg vil i det følgende se nærmere på, hvordan politikere har italesat ungdomsuddannelserne i folketingssalen.
Jeg vil særligt gerne undersøge hvordan  erhvervsuddannelserne omtales.\todo(er dette min PF?)
Hvilke udsnit af de unge har man tænkt, at sortere i EUD-kassen?
Hvordan omtales de set i forhold til de gymnasiale ungdomsuddannelser? \todo{hvad er acccpteredt shorthand?}

ved hjælp af computeranalyser af folketingsmøder fra 1953 til i dag.
Det vil ikke være en decideret historisk gennemgang, men jeg vil dog give et kort historisk oprids af reformer indenfor ungdomsudannelserne, for at hae en kontekst for den omtale, jeg ser i folketingssalen.
Jeg konkluderer specialet i del V, med en diskussion af mine resultater, og hvad betydningen kan være for videre forskning.

95 \perc af alle ungdomsårgange på ungdomsuddannelser i 90erne, revideret med reform 2000 til, at 97?? skulle gennemføre

'ny nordisk skole' fadæsen - den levede ikke ret længe
