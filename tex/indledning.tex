\section{Indledning}

citat fra politiker ca. 2000 - vidensamfundet!

citat fra politiker ca 2016 - håndværkere!

Ingen politiker kan undlade at have en holdning til uddannelse.
Fra børnehaven, der er blevet mere eksplicit skoleforberedende, hele vejen til universitetet, der skal imødekomme en verden i global konkurrence.\todo{citation needed}
Jeg akter at gå i dybden omkring ungdomsuddannelserne, hvor der indenfor de seneste 25 år har været 7\todo{mindst!} reformer af forskellig grad.\todo{tjek tal, find kilde }

Ungdomsuddannelserne har også mange opgaver at løse i dag.
De skal på samme tid
\begin{itemize}
  \item
    sikre, at Danmark har den kvalificerede arbejdskraft der er behov for, nu og i fremtiden
  \item
    sikre, at Danmark kan begå sig i den internationale konkurrence
  \item
    sikre social mobilitet, således at alle der ønsker en ungdoms- eller videregående uddannelse har mulighed for at få en
  \item
    sikre den danske velfærdsstats overlevelse, ud fra devicen, at uddannelse vil give bedre mulighed for arbejde; og dermed selvforsørgelse \todo{omskrives?}
  \item
    sikre sammenhængskraften i det danske samfund, hvor man vil undgå, at der er grupperinger der er “udenfor” fællesskabet\footnote{Mangel på uddannelse er her set som en proxy-variabel for forskellige grader af “social udsathed” — levealder, overførselsindkomst, livstilssygdomme mm.}\todo{citation needed}
  \item
    bidrage til den (fortsatte) dannelse af de unge, til oplyste og refleksive nations- og verdensborgere\todo{citation needed}
\end{itemize}

Flere af de ovenstående formål griber ind i hinanden, og bringer stikord som “omfordelingspolitik”, “human capital” og “globalisering” frem.\todo{that is ikke a good vending}
Så langt, så godt.
Hvad angår udtalte mål, i hvert fald.
Der er dog nogle mere skjulte formål i uddannelsessystemet.
Uddannelsessystemet er de facto en sorteringsmekanisme - det er det til en vis grad nødt til at være, jævnfør de to første elementer i listen ovenfor.
Men hvad angår den sociale mobilitet, er der også elementer af, en legitimering af strukturelle forskelle \todo{citation needed}.
Dette forstærkes (muligvis) af en form for privilegieblindhed i Danmark.  \todo{skal uddybes, og underbygges}
Der er en udbredt forståelse af Danmark som et forholdsvist fladt samfund, hvor “alle” har mulighed for at blive ”alt”, hvilket kan gøre, at de dokumenterede forskelle i uddannelsesudfald ikke tages alvorligt.\todo{shots fired. need cover}

Som citaterne overfor viser, varierer de politiske holdninger og udsagn om ungdomsuddannelserne og deres formål.

I det første citat ser vi....
Hvilket udmøntede sig i KONKRET UDD POL UDSPIL X

I det andet citat ser vi...
Dette endte i KONKRET UDD POL UDSPIL Y

Jeg vil grave dybere i dette emne.
Jeg vil specifikt se på, hvordan politikerene taler om ungdomsuddannelser i folketingssalen, ved hjælp af computeranalyser af folketingsmøder fra 1953 til i dag.

hurtigstartsbonus 2009 → afskaffet 2018, med virkning 2020
nsad
fremdriftsreform 2013 → hurtiogere færdig med vidergående uddannelser


95 \perc af alle ungdomsårgange på ungdomsuddannelser i 90erne, revideret med reform 2000 til, at 97?? skulle gennemføre
