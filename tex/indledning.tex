\section{Indledning}\label{sec:intro}
\begin{epigraphs}
\qitem{\itshape
De unge stiller i dag store krav og forventninger om udviklingsmuligheder.
Det er samfundets opgave at reagere imødekommende på disse behov.
Hver enkelt af de unge skal have
reelle muligheder for at opnå de bedst mulige uddannelsesmæssige forudsætninger for et rigt voksenliv.
Det er i hele samfundets interesse at gøre en forstærket indsats.
At efterlade en stor del af en ungdomsgeneration uden de nødvendige kvalifikationer er at lade dem i stikken i en fremtid, der stiller stadig større krav til uddannelse.
En stor gruppe unge uden uddannelse er måske frem for alt et problem for demokratiet.
}
{Undervisningsminister Ole Vig Jensen, Redegørelse til folketinget om uddannelse til alle, 11. november \citeyear{jensenRedegorelseR319931993}}

\qitem{\itshape
Vi skal have uddannelser i verdensklasse. Målet er, at eleverne i folkeskolen bliver blandt verdens bedste til læsning, matematik og naturfag. At mindst 95 pct. af de unge får en ungdomsuddannelse i 2015. Og at mindst 50 pct. af en ungdomsårgang får en videregående uddannelse i 2015.
}
{Statsminister Anders Fogh Rasmussen, tale ved folketingets åbning, 24. februar \citeyear{rasmussenStatsministerAndersFogh2005}}
\end{epigraphs}

 Om der er fokus på, at klare sig i den internationale konkurrence eller, at få “alle” med; så er der få politikere, der kan undlade at have en holdning til uddannelse.

Undervisningsminister Ole Vig Jensen funderer for eksempel over, hvorvidt de u(ud)dannede unge kan udgøre en trussel mod demokratiet.
Dette er ikke en ny tanke. Siden 1970erne har den såkaldte “restgruppe” vagt politisk bekymring \autocite{hansenRestgruppen2004}.
Hvad gør man lige med den (forholdsvis stabile) andel af en ungdomsårgang, der ikke har kompetencegivende uddannelse, i et arbejdsmarked der har stadig mere rigide krav om formelle kvalifikationer \autocite{dpuNyhedsbrevUddannelseKan2001}?

 Vel et årti senere lover statsminister Anders Fogh Rasmussen uddannelse “i verdensklasse”. “Restgruppeproblemet” er langt fra løst, men nu er fokusset den globale konkurrence.
 Her skal Danmark hævde sig, og i bedste human capital stil, skal de unges potentialer forløses.

\subsection{Problemfelt}\label{sec:problem}
Der ligger dog mere under overfladen, når man taler om “uddannelse” end en maskine, hvor man kan skrue op for “integration” og “innovation”.

Uddannelsessystemet fungerer, for eksempel, de facto som en sorteringsmekanisme.
For at illustrere dette, er der i det følgende uddrag fra nogle formålsparagraffer for forskellige ungdomsuddannelser:

\blockquote[\citetitle{uddannelsesministerietBekendtgorelseAfLov2016a},  \cite{uddannelsesministerietBekendtgorelseAfLov2016a}]{
§ 1. Formålet med erhvervsgrunduddannelsen er, at den unge opnår personlige, sociale og faglige kvalifikationer, som dels giver umiddelbar adgang til at fortsætte i en erhvervskompetencegivende uddannelse, dels giver grundlag for beskæftigelse på arbejdsmarkedet. 
Erhvervsgrunduddannelsen skal tillige bidrage til at udvikle den unges interesse for og evne til aktiv medvirken i et demokratisk samfund.
}
\blockquote[\citetitle{uddannelsesministerietBekendtgorelseAfLov2020},  \cite{uddannelsesministerietBekendtgorelseAfLov2020}]{
Stk. 2. Dette uddannelsessystem skal tilrettelægges således, at det i videst muligt omfang er egnet til at
\begin{enumerate}
  \item
    motivere til uddannelse og sikre, at alle, der ønsker en erhvervsuddannelse, får reelle muligheder herfor og for at vælge inden for en større flerhed af uddannelser,

  \item
    give uddannelsessøgende en uddannelse, der giver grundlag for deres fremtidige arbejdsliv, herunder etablering af selvstændig virksomhed,

  \item
    bidrage til at udvikle de uddannelsessøgendes interesse for og evne til aktiv medvirken i et demokratisk samfund og bidrage til deres personlige udvikling, karakterdannelse og faglige stolthed
\textins[\ldots]
\end{enumerate}
}
\blockquote[\citetitle{uddannelsesministerietBekendtgorelseAfLov2019}, \cite{uddannelsesministerietBekendtgorelseAfLov2019}]{

Stk. 2. Eleverne skal gennem uddannelsens faglige og pædagogiske progression udvikle faglig indsigt og studiekompetence. De skal opnå fortrolighed med at anvende forskellige arbejdsformer og opnå evne til at fungere i et studiemiljø, hvor kravene til selvstændighed, samarbejde og sans for at opsøge viden er centrale.

Stk. 3. Uddannelserne skal have et dannelsesperspektiv med vægt på elevernes udvikling af personlig myndighed. Eleverne skal derfor lære at forholde sig reflekterende og ansvarligt til deres omverden: medmennesker, natur og samfund samt til deres udvikling. Uddannelserne skal tillige udvikle elevernes kreative og innovative evner og kritiske sans.

Stk. 4. Uddannelserne og institutionskulturen som helhed skal forberede eleverne til medbestemmelse, medansvar, rettigheder og pligter i et samfund med frihed og folkestyre. Undervisningen og hele institutionens dagligliv må derfor bygge på åndsfrihed, ligeværd og demokrati og styrke elevernes kendskab til og respekt for grundlæggende friheds- og menneskerettigheder, herunder ligestilling mellem kønnene. Eleverne skal derigennem opnå forudsætninger for aktiv medvirken i et demokratisk samfund og forståelse for mulighederne for individuelt og i fællesskab at bidrage til udvikling og forandring samt forståelse af såvel det nære som det europæiske og det globale perspektiv.
}

De tre ungdomsuddannelser --- erhvervsgrunduddannelsen, erhvervsuddannelserne og de gymnasiale uddannelser —  har særskilte mål. Nogle skal enten i uddannelse eller i arbejde, andre skal uddannes til arbejdsmarkedet, og andre igen skal forberedes til universitetet.

Ud fra forståelser af det arbejdsdelte samfund giver dette god mening.
Vi kan ikke alle være historikere, læger eller sociologer - samfundet har også behov for pædagoger, salgsassistenter og murere.
Hvis uddannelsessystemet har som funktion, at sortere borgermassen i differentierede arbejdsområder, er der også elementer af, en legitimering af strukturelle forskelle \todo{kilde: legitimering af strukturelle forskelle}.
Denne skules dog af en forståelse af Danmark som et forholdsvist fladt samfund, hvor “alle” har mulighed for at blive “alt”, til trods for, at der dokumenterede forskelle i uddannelsesudfald på baggrund af socioøkonomisk udgangspunkt.\todo{kilde:  det socialt flade danmark}



\subsection{Sortering til ungdomsuddannelserne i dag}
\label{sec:sorting}
Der er en faldende tilslutning til erhvervsuddannelserne.
Tal fra \citeauthor{borne-ogundervisningsministerietSogning} viser, at andelen unge, der søger erhvervsuddannelser efter endt grundskole er faldet jævnt fra godt 31 procent i 2001, til knap 20 procent 2013, hvor den har været nogenlunde stabil.
Der har været en tilsvarende stigning i ansøgere til de gymnasiale uddannelser, der så en stigning fra knap 60 procent til omkring 74 procent i samme periode \autocite[s 5f]{undervisningsministerietog10klasseelevernes2017}.
Holdt sammen med, at frafaldet er markant højere på erhvervsuddannelserne end de gymnasiale uddannelser \autocite{danskegymnasierFuldforelseOgKarakterer2019}, er politiske målsætninger om, at flere gennemfører en ungdomsuddannelse tilsyneladende langt fra at være opnåede.
Der er blandt andet bekymringer om, man kan samle “restgruppen” op.
Andelen unge uden kompetencegivende uddannelse eller tilknytning til arbejdslivet har således ligget stabilt på omkring 7-8 procent i perioden 2012-2017 \autocite[s. 9]{andersenUngeUdenUddannelse2019}.

Qua de strukturelle uligheder der er nævnt tidligere, er der også forskelle i prestige og social agtelse i forskellige uddannelser.
Jeg henviser igen til uddragene fra formålsparagrafferne ovenfor.
Det lyder måske “flottere”, at skulle “lære at forholde sig reflekterende og ansvarligt til deres omverden: medmennesker, natur og samfund samt til deres udvikling” end at skulle udvikle “interesse for og evne til aktiv medvirken i et demokratisk samfund”.
Men der er til gengæld heller ikke alle, der ser frem til 5-10 års yderligere skolegang efter grundskolen.
Disser 

\todo{uddybe social agtelse/kapital?}

Hvordan man helt præcis stiller sorteringsmekanismerne i uddannelse ind, kan man (blive ved med) at skrive meget om.
Hvordan kan det være, at restgruppen ikke er blevet minimeret over ?? år, trods ihærdige initiativer?
Kan den meget tydelige kønssegregation i Danmarks uddannelsessystem og arbejdsmarked imødekommes?
\footnote{Dette ser man i øvrigt også ved ansøgning til erhvervsuddannelserne — drengene søger indenfor byggeri og teknik, pigerne søger ind på sundhed og service} \todo{kilde: køn og EUD}.
Hvordan forholder de unge sig til valget af ungdomsuddannelse?
Er danmark ved at være overuddannet, som formændene for Dansk Metal og HK udtalte for nogle år siden? \autocite{simonsenLadOsGore2016}
En dybdegående analyse af “Uddannelse til alle” ville være grundlag for et speciale i sig selv.

Jeg er dog mere interesseret i, hvordan politikere taler om uddannelse.
Hvordan har de folkevalgte italesat deres uddannelsespolitiske holdninger gennem tiden?
Til mit held, er referater fra møder i folketingssalen offentligt tilgængelige, for frit brug.
Disse tekster vil jeg foretage computerbaserede tekstanalyser på.
Computerbaserede analyser er langt fra fejlbarlige, men det er dog meget nemmere for en computer at se på over 800,000 taler end det er at læse dem en efter en.

\subsection{Problemformulering}\label{sec:pf}
Jeg vil i dette speciale se nærmere på, hvordan politikere har italesat uddannelse i folketingssalen, ved hjælp af computeranalyser af folketingsmøder fra 1953 til i dag.
Jeg vil have ungdomsuddannelserne i fokus for min undersøgelse, herunder især den politiske diskurs omkring de erhvervsrettede ungdomsuddannelser.
Jeg vil også berøre de gymnasiale uddannelser, som kontrast og modpol til den politiske omtale af EUD.\todo{formulering: GYM vs EUD}

\subsubsection{Undersøgelsesspørgsmål}\label{sec:resqs}

Hvilke omtaler af erhvervsuddannelserne er kendetegnende for specifikke perioder?

Hvor optræder der eventuel politisk enighed eller uenighed?

Kan de politiske udtalelser omkring uddannelse gøres op efter partipolitiske linjer?

Hvem ser politikerne som målgruppen for de erhvervsfaglige uddannelser?

Hvordan positionerer politikerne sig og andre i kraft af sine talehandlinger?

\subsection{Specialets videre opbygning}\label{sec:structure}

Del II vil, efter en gennemgang af anden forskning i de danske erhvervsuddannelser, kort omhandle historiske tendenser i dansk uddannelsespolitk.
For at have en historisk kontekst for min analyse, agter jeg at gå mere i dybden hvad angår de erhvervsrettede ungdomsuddannelser, med et oprids over reformer og ændringer fra 1953 til i dag.\todo{David: Giver det mening?}

I del III vil jeg beskrive mine metodiske og teoretiske overvejelser.
Styrker og svagheder ved computerbaserede tekstanalyser vil blive gennemgået, og de konkrete metoder beskrevet.
Jeg vil også drøfte sociologiske ståsteder\todo{David: hvilken teori? er meget åben} der kan underbygge min analyse, herunder positioneringsteori\todo{David: Denne er valgt da den konkret ser på talehandlinger som konstruerende for og konstrueret af den sociale virkelighed}, samt drøftelser omkring uddannelsens betydning for videreførelse af sociale strukturer.

\todo{Analyse og konkl uddybes med tiden}
I del IV findes en gennemgang af mine analyser.

Jeg afslutter specialet i del V, med en diskussion af mine resultater, og hvad betydningen kan være for videre forskning.
