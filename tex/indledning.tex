\section{Indledning}

citat fra politiker ca. 2000 - vidensamfundet!

citat fra politiker ca 2016 - håndværkere!

Ingen politiker kan undlade at have en holdning til uddannelse.
Fra børnehaven, der er blevet mere eksplicit skoleforberedende, hele vejen til universitetet, der skal imødekomme en verden i global konkurrence.\todo{citation needed}
Jeg akter at gå i dybden omkring ungdomsuddannelserne, hvor der indenfor de seneste 25 år har været 7\todo{mindst!} reformer af forskellig grad.\todo{tjek tal, find kilde }

Ungdomsuddannelserne har også mange opgaver at løse i dag.
De skal på samme tid
\begin{itemize}
  \item
    sikre, at Danmark har den kvalificerede arbejdskraft der er behov for, nu og i fremtiden
  \item
    sikre social mobilitet, således at alle der ønsker en ungdoms- eller videregående uddannelse har mulighed for at få en
  \item
    sikre den danske velfærdsstats overlevelse, ud fra devicen, at uddannelse vil give bedre mulighed for arbejde; og dermed selvforsørgelse \todo{omskrives?}
  \item
    sikre sammenhængskraften i det danske samfund, hvor man vil undgå, at der er grupperinger der er “udenfor” fællesskabet\footnote{Mangel på uddannelse er her set som en proxy-variabel for forskellige grader af “social udsathed” — levealder, overførselsindkomst, livstilssygdomme mm.}\todo{citation needed}
  \item
    bidrage til den (fortsatte) dannelse af de unge, til oplyste og refleksive nations- og verdensborgere\todo{citation needed}
\end{itemize}

Flere af de ovenstående formål griber ind i hinanden, og bringer stikord som “omfordelingspolitik”, “human capital” og “globalisering” frem.

(noget med legitmering af sociale forskelle og strukturelle fordelingsmekanismer, gerne med henvisning til “raceblindhed”)

Men, som citaterne overfor viser, svinger de politiske holdninger til ungdomsuddannelserne og deres formål.


Jeg vil specifikt se på, hvordan politikerene taler om ungdomsuddannelser i folketingssalen, ved hjælp af computeranalyser af folketingsmøder fra 1953 til i dag.

hurtigstartsbonus 2009 → afskaffet 2018, med virkning 2020
nsad
fremdriftsreform 2013 → hurtiogere færdig med vidergående uddannelser


95 \perc af alle ungdomsårgange på ungdomsuddannelser i 90erne, revideret med reform 2000 til, at 97?? skulle gennemføre
