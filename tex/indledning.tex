\part{Indledning}\label{part:intro}
\begin{epigraphs}
\qitem{\itshape
De unge stiller i dag store krav og forventninger om udviklingsmuligheder.
Det er samfundets opgave at reagere imødekommende på disse behov.
Hver enkelt af de unge skal have
reelle muligheder for at opnå de bedst mulige uddannelsesmæssige forudsætninger for et rigt voksenliv.
Det er i hele samfundets interesse at gøre en forstærket indsats.
At efterlade en stor del af en ungdomsgeneration uden de nødvendige kvalifikationer er at lade dem i stikken i en fremtid, der stiller stadig større krav til uddannelse.
En stor gruppe unge uden uddannelse er måske frem for alt et problem for demokratiet.
}
{Undervisningsminister Ole Vig Jensen, Redegørelse til folketinget om uddannelse til alle, 11. november \citeyear{jensenRedegorelseR319931993}}

\qitem{\itshape
Vi skal have uddannelser i verdensklasse. Målet er, at eleverne i folkeskolen bliver blandt verdens bedste til læsning, matematik og naturfag. At mindst 95 pct. af de unge får en ungdomsuddannelse i 2015. Og at mindst 50 pct. af en ungdomsårgang får en videregående uddannelse i 2015.
}
{Statsminister Anders Fogh Rasmussen, tale ved folketingets åbning, 24. februar \citeyear{rasmussenStatsministerAndersFogh2005}}
\end{epigraphs}

\chapter{Indledning}\label{chap:intro}

 Om der er fokus på, at klare sig i den internationale konkurrence eller, at få “alle” med; så er der få politikere, der kan undlade at have en holdning til uddannelse.

Undervisningsminister Ole Vig Jensen funderer for eksempel over, hvorvidt de u(ud)dannede unge kan udgøre en trussel mod demokratiet.
Dette er ikke en ny tanke. Siden 1970erne har den såkaldte “restgruppe” vagt politisk bekymring \autocite{hansenRestgruppen2004}.
Hvad gør man lige med den (forholdsvis stabile) andel af en ungdomsårgang, der ikke har kompetencegivende uddannelse, i et arbejdsmarked der har stadig mere rigide krav om formelle kvalifikationer \autocite{dpuNyhedsbrevUddannelseKan2001}?

 Vel et årti senere lover statsminister Anders Fogh Rasmussen uddannelse “i verdensklasse”. “Restgruppeproblemet” er langt fra løst, men nu er fokusset den globale konkurrence.
 Her skal Danmark hævde sig, og i bedste human capital stil, skal de unges potentialer forløses.

 Ungdomsuddannelserne danner rammen for en af de vigtigste overgange i et menneskes liv.
 Man lægger barndommen fra sig i folkeskolen; og tager fat på det særligt moderne fænomen “ungdom” på vejen mod voksenlivet.
 \citeauthor{juulDiskurserOmUngdom2013} beskriver hvordan modernitetens ungdom er i refleksiv (re)konstruktion som individer.
 Samtidigt er ungdommen underlagt sociale strukturer, der begrænser deres reelle muligheder for at realisere et selvbiografisk projekt (\citeyear[s. 11]{juulDiskurserOmUngdom2013}).

I \citetitle{juulDiskurserOmUngdom2013} beskæftiger \citeauthor{juulDiskurserOmUngdom2013} sig med ungdomspolitik i det seneste århundrede, og de diskurser om ungdomslivet man kan udlede heraf.
Hun henviser bl.a til at man i 70'erne havde fokus på, hvordan statens institutioner kunne havde indflydelse på de unges liv; hvor fokuset i 90erne var flyttet over mod overgangen til arbejdslivet.
Eller, en sociologisk inspireret ungdomsidskurs underordnes en (samfunds)økonomisk forståelsesramme.

Jeg vil i dette speciale grave dybere ned i ungdomsuddannelserne og de politiske holdninger herfor, og hvordan disse eventuelt har ændret karakter over tid siden 1978.
I det jeg ser på ungdomsuddannelserne, vil jeg have et særligt fokus på erhvervsuddannelserne, og hvordan disse vægtes i forhold til de akademisk rettede ungdomsuddannelser.

For at kunne undersøge dette felt, vil jeg tage referater fra møderne i folketingssalen i brug.
Da dette er et overvældende stort kildemateriale\footnote{Omkring 740.000 taler som udgangspunkt}, at analysere og klassificere for egen hånd, vil jeg tage computerbaserede tekstanalyseværktøjer i brug.
Dette speciale vil dermed også fremstå som en afprøvning af tekstmining og lignende analyseværktøjers anvendelighed i dansk pædagogisk-sociologisk sammenhæng.

\section{At beskrive kontra at forklare}\label{sec:beskrive}

Dette speciale lægger sig i en beskrivende sociologisk tradition.
Jeg håber, at kunne påvise nogle mønstre i dansk uddannelsespolitik, men jeg vil ikke umiddelbart efterstræbe en forklaringsmodel eller forsøge at påvise kausalitet ud fra mine observationer\footnote{Implicit i denne vinkling er et ontologisk og epistemologisk perspektiv, der tilsiger at der er en social verden at beskrive — og at en sådan beskrivelse kan afspejle en social virkelighed}.

Der er dog en vis spænding mellem det deskriptive og det kausale.
Allerede i \citeyear{sjobergRationaleDescriptiveSociology1951} så \citeauthor{sjobergRationaleDescriptiveSociology1951} det nødvendigt, at tage den beskrivende sociologi i forsvar \autocite{sjobergRationaleDescriptiveSociology1951}.
En renskåret afgrænsning mellem beskrivelse og forklaring kan endda anfægtes; i det dybdegående og systematiske beskrivelser vil kunne have en forklarende kraft \autocite[s. 252]{sjobergRationaleDescriptiveSociology1951}.

I nærmere fortid kan nævnes \citeauthor{savageContemporarySociologyChallenge2009}, der i \citeyear{savageContemporarySociologyChallenge2009} argumenterede for, at det deskriptive er en central del af megen moderne sociologi, bredt antaget \autocite{savageContemporarySociologyChallenge2009}.
Dette især, hvor det empiriske materiale sociologer beskæftiger sig med ikke nødvendigvis opstår som del af sirligt planlagte spørgeskemaer eller længerevarende feltstudier \autocite[s. 157]{savageContemporarySociologyChallenge2009} — som for eksempel er gældende for dette speciale.

Jeg orienterer mig både indenfor den pædagogiske sociologi, den politiske sociologi og en nyere “algoritmisk” sociologisk tilgang.
Både \citeauthor{sjobergRationaleDescriptiveSociology1951} og \citeauthor{savageContemporarySociologyChallenge2009} ser en værdi i, at beskrivelser kan indeholdes i flere metodologiske referencerammer \autocite[s. 256]{sjobergRationaleDescriptiveSociology1951} eller endda opfordrer til en gentænkning af klassiske kategoriskeringer af moderne sociologisk gerning \autocite[s. 170]{savageContemporarySociologyChallenge2009}.
\citeauthor{savageContemporarySociologyChallenge2009} mener videre at se, at denne nye dragning til det deskriptive fører til en orientering mod det visuelle i formidlingen \autocite[s.169]{savageContemporarySociologyChallenge2009}.
I arbejdet med at præsentere mine data og metoder, kan jeg anekdotisk sige, at denne observation holder for denne opgaves vedkommende.

\chapter{Problemfelt}\label{sec:problem}

De indledende citater for specialets befinder sig begge i et overordnet neoliberalt paradigme for samfundets opbygning; om end med forskelligt fokus.
Det er antaget skadeligt for hele samfundet, hvis vi ikke alle har mulighed til, at yde deres ypperste.

Der ligger dog mere under overfladen, når man taler om “uddannelse” end en maskine, hvor man kan skrue op for “integration” og “innovation”.
Det (post)moderne samfund er tiltagende specialiseret og arbejdsdelt \autocite{baumanLiquidModernity2000}.
Ud fra dette perspektiv giver dette god mening, at der skal etableres nogle rammer for det selvrealiseringsprojekt der blev henvist til af \citeauthor{juulDiskurserOmUngdom2013} ovenfor.
Vi kan ikke alle være historikere, læger eller sociologer — samfundet har også behov for pædagoger, salgsassistenter og murere.
Uddannelsessystemet fungerer i denne sammenhæng de facto som en sorteringsmekanisme; der sluser de unge mod de funktioner, de kan tjene samfundet i.

Hvis uddannelsessystemet har som funktion, at sortere borgermassen i differentierede arbejdsområder, vil der også elementer af, en legitimering af strukturelle forskelle.
Meget af sproget i og omkring ungdomsuddannelserne handler om, at imødekomme de unges (antagne) forudsætninger og interesser.
En opdeling af de unge i forskellige spor — på engelsk “tracking” — er ofte set som en vej til, at opnå disse (formodede) fordele.
Dermed kan de “bogligt interesserede” søge mod de akademiske ungdomsuddannelser; og de mere “praktisk orienterede” har muligheder indenfor erhvervsuddannelserne.
Dog viser internationale undersøgelser, at tildeling til de forskellige spor i vid udstrækning er stærkt associeret med øvrige klasseskæl i samfundet; og at elever i højere spor øger deres forspring set i et livslangt perspektiv \autocite[s. 3]{gamoranTrackingInequalityNew2010}.

Denne grundlæggende forskelsbehandling i praksis er dog skjult bag en forståelse af Danmark som et forholdsvist fladt samfund — hvor man har meget få begrænsninger i udarbejdelsen af sin selvbiografi.
“Alle” har (principielt) mulighed for at blive “alt”, til trods for, at der dokumenterede forskelle i uddannelsesudfald på baggrund af socioøkonomisk udgangspunkt \autocite[se fx s 195ff i]{munkSocialUlighedOg2014}.
Som \citeauthor{munkSocialUlighedOg2014} beskriver, er uddannelsesmobiliteten godt nok steget i de seneste år.
Familiebaggrund betyder nu mindre end før, men viser sig fx i hvilke videregående uddannelser unge fra (historisk) mindre priviligerede samfundsgrupper søger mod (\citeyear[s. 199]{munkSocialUlighedOg2014}).

\section{Sortering til ungdomsuddannelserne i dag}\label{chap:sorting}

Der er en faldende tilslutning til erhvervsuddannelserne blandt de unge.
Tal fra \citeauthor{borne-ogundervisningsministerietHvemOgHvor} viser, at andelen unge, der søger erhvervsuddannelser efter endt grundskole er faldet jævnt fra godt 31 procent i 2001, til knap 20 procent 2013, hvor den har været nogenlunde stabil \autocite[s. 6]{borne-ogundervisningsministerietHvemOgHvor}.
Der har været en tilsvarende stigning i ansøgere til de gymnasiale uddannelser, der så en stigning fra knap 60 procent til omkring 74 procent i samme periode \autocite[s 5f]{undervisningsministerietOg10Klasseelevernes2017}.
Holdt sammen med, at frafaldet er markant højere på erhvervsuddannelserne end de gymnasiale uddannelser \autocite{danskegymnasierFuldforelseOgKarakterer2019}, er politiske målsætninger om, at flere gennemfører en ungdomsuddannelse tilsyneladende langt fra at være opnåede.
Der er blandt andet bekymringer om, man kan samle “restgruppen” op.
Andelen unge uden kompetencegivende uddannelse eller tilknytning til arbejdslivet har således ligget stabilt på omkring 7-8 procent i perioden 2012-2017 \autocite[s. 9]{andersenUngeUdenUddannelse2019}.

Qua de strukturelle uligheder der er nævnt tidligere, er der også forskelle i prestige og social agtelse i forskellige uddannelser.Denne manglende sociale agtelse underbygges af en opgørelse fra \citeauthor{danmarksstatistikErhvervsuddannelserDanmark20192019}, hvor kun 60 procent af befolkningen finder, at de danske erhvervsuddannelser har et positivt omdømme (\citeyear[s. 7]{danmarksstatistikErhvervsuddannelserDanmark20192019}).

Det er til gengæld heller ikke alle, der nødvendigvis ser frem til 5-10 års yderligere skolegang efter grundskolen, hvor en mere direkte vej til arbejdsmarkedet er oplagt.
Men også her er der en social skævvridning.
Som samme opgørelse fra \citeauthor{danmarksstatistikErhvervsuddannelserDanmark20192019} viser, har for eksempel 19 procent af eleverne på erhvervsuddannelse forældre med en årlig bruttoindkomst på under 300 000, modsat kun 6 procent af eleverne på de gymnasiale uddannelser (\citeyear[s. 6]{danmarksstatistikErhvervsuddannelserDanmark20192019}). 

Denne forskel i social agtelse kan også ses, i hvordan formålsparagrafferne for tre forskellige ungdomsuddannelser italesætter de unge. Ved at se på erhvervsgrunduddannelsen \autocite[§ 1]{uddannelsesministerietBekendtgorelseAfLov2016a}, erhvervsuddannelserne \autocite[§1, stk.2]{uddannelsesministerietBekendtgorelseAfLov2020} og de gymnasiale uddannelser \autocite[§1]{uddannelsesministerietBekendtgorelseAfLov2019} viser, særskilte forestillinger af forskellige slags ungdom.

EGU-eleven skal opnå (eller sågar tilføres) kvaliteter, denne dermed må have som mangelvare; EUD-eleven skal understøttes i, at følge sit eget initiativ; og gymnasie-eleven fremstår enestående som demokratisk samfundsborger \textit{in spe} — der skal blot lidt understøttet (selv)udvikling til.
Disse fremstillinger af dansk ungdom skal henholdsvis i uddannelse eller i arbejde; uddannes til arbejdsmarkedet; og andre igen skal forberedes til universitetet.

Hvordan man helt præcis stiller sorteringsmekanismerne i uddannelse ind, kan man (blive ved med) at skrive meget om:
\begin{itemize}
  \item
    Hvordan kan det være, at restgruppen ikke er blevet betydeligt minimeret de seneste år, trods ihærdige initiativer?
  \item
    Kan den meget tydelige kønssegregation i Danmarks uddannelsessystem og arbejdsmarked imødekommes\footnote{Dette ser man i øvrigt også ved ansøgning til erhvervsuddannelserne — drengene søger overordnet ind på fx byggeri og teknik, pigerne søger ind på sundhed og service \autocite[s. 48]{danmarksstatistikErhvervsuddannelserDanmark20192019}}.
  \item
    Hvordan forholder de unge sig til valget af ungdomsuddannelse?
  \item
    Er danmark ved at være overuddannet, som formændene for Dansk Metal og HK udtalte for nogle år siden? \autocite{simonsenLadOsGore2016}
\end{itemize}

En dybdegående analyse af “Uddannelse til alle” i 90erne og fremefter ville forsåvidt være grundlag for et speciale i sig selv.

Jeg er dog mere interesseret i, hvordan politikere \textit{taler om} uddannelse.
Hvordan har de folkevalgte italesat deres uddannelsespolitiske holdninger gennem tiden?
Til mit held, er referater fra møder i folketingssalen offentligt tilgængelige, for frit brug.
Her kan man se hvordan politikere henvender sig til andre politikere — samtidig med, at de også taler i en semi\ offentlig kontekst.
Disse tekster vil jeg foretage computerbaserede tekstanalyser på.
Computerbaserede analyser er langt fra ufejlbarlige, men det er dog meget nemmere for en computer at se på over 700,000 taler end det er at læse dem en efter en.

\chapter{Problemformulering}\label{chap:pf}
Jeg vil i dette speciale se nærmere på, hvordan politikere har italesat ungdomsuddannelse i folketingssalen.
Herunder især den politiske diskurs omkring de erhvervsrettede ungdomsuddannelser.
Jeg vil også berøre de gymnasiale uddannelser, og hvordan debatten mellem “arbejdsmarkedsrelaterede kompetencer” og såkaldt “overuddannelse” arter sig.
Min undersøgelse vil tage udgangspunkt i computeranalyser af politiske taler ved møder i folketingssalen fra 30. august 1978 til i dag.

Valget af 1978 som udgangspunkt er dels på baggrund af datatilgængelighed.
Men 1978 er også anvendelig som et nedslagspunkt i dansk ungdomspolitik og uddannelseshistorie.
U90 blev offentliggjort i 1978, med visioner om hvordan dansk uddannelse skulle se ud mod 1990 \autocite{undervisningsministeriet90SamletUddannelsesplanlaegning1978}
Derudover blev den erhvervsfaglige grunduddannelse vedtaget nationalt i 1977 \autocite{thewikipediavolunteersEFG2019}, og jeg kan dermed få diskurserne efterfølgende taget med.
Den specifikke dato — 30. august 1978 — er den første dag for regeringen Anker Jørgensen III.
Ved at holde mig til regeringsperioder, kan jeg holde en ekstra analysedimension — er taleren i regering eller i opposition? — åben, hvis dette skulle være relevant.

\section{Undersøgelsesspørgsmål}\label{seq:resqs}

\begin{itemize}
  \item
    Hvilke omtaler af erhvervsuddannelserne er kendetegnende for specifikke perioder?
  \item
    Hvor optræder der eventuel politisk enighed eller uenighed?
  \item
    Kan de politiske udtalelser omkring uddannelse gøres op efter partipolitiske linjer?
  \item
    Hvem ser politikerne som målgruppen for de erhvervsfaglige uddannelser?
  \item
    Hvordan positionerer politikerne sig og andre i kraft af sine talehandlinger?
  \item
    Kan man se tendenser til, at genskabe og legitimere samfundets strukturforskelle?
\end{itemize}

Disse spørgsmål bevæger sig fra det forholdsvis konkrete til det mere abstrakte.
Jeg er noget nysgerrig omkring hvorvidt de redskaber jeg vil tage i brug kan bidrage til indsigter på det mere abstrakte niveau.

\chapter{Specialets opbygning}
I denne første del af specialet har jeg præsenteret mit undersøgelsesobjekt og dettes relevans i en pædagogisk sociologisk kontekst, med en yderligere præsentation af min undersøgelsestilgang og mine underspørgsmål.

i specialets næste del vil jeg bestræbe mig på, at give baggrund og kontekst for min analyse.
Jeg vil gennemgå et uddrag af senere tids forskning i ungdoms\ uddannelser, med et særligt øje for erhvervsuddannelserne.
For erfaringer med computeranalyser af parliamentarisk materiale, drager jeg på et overvejende politisk-sociologisk forskningsarbejde.
For at have en historisk kvalifikation for min analyse, og min inddeling i perioder til sammenligning, agter jeg derudover at give et kort oprids over tendenser i dansk uddannelses\ politik de seneste 4 årtier.
Også her vil jeg særligt skele til ungdomsuddannelserne, herunder erhvervsuddannelserne.

I del III vil jeg drøfte mine metodiske og teoretiske overvejelser.
Styrker og svagheder ved computerbaserede tekstanalyser vil blive gennemgået. Jeg vil derefter dykke ned mine konkrete metoder og overvejelser set i forhold til det gældende dataset og mine problemstillinger.

En gennemgang af min analyse vil være fokus for del IV.
Jeg vender tilbage til mine undersøgelsesspørgsmål, og mit arbejde med, at besvare disse.
Her vil jeg illustrere og beskrive mine resultater (eller mangel på samme), og hvordan jeg er kommet frem til disse.

Jeg afslutter specialet i del V, med en diskussion af mine resultater, og hvad betydningen kan være for vores forståelse af uddannelsespolitik og ungdomsudannelser. Jeg vil også drøfte implikationer for fremtidige undersøgelser i spændingsfeltet politisk og pædagogisk sociologi.
