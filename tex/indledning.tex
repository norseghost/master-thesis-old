\section{Indledning}
\label{sec:intro}
\begin{epigraphs}
\qitem{
De unge stiller i dag store krav og forventninger om udviklingsmuligheder.
Det er samfundets opgave at reagere imødekommende på disse behov.
Hver enkelt af de unge skal have
reelle muligheder for at opnå de bedst mulige uddannelsesmæssige forudsætninger for et rigt voksenliv.

Det er i hele samfundets interesse at gøre en forstærket indsats.
At efterlade en stor del af en ungdomsgeneration uden de nødvendige kvalifikationer er at lade dem i stikken i en fremtid, der stiller stadig større krav til uddannelse.
En stor gruppe unge uden uddannelse er måske frem for alt et problem for demokratiet.
}
{Undervisningsminister Ole Vig Jensen, Redegørelse til folketinget om uddannelse til alle, 11. november 1993}

\qitem{
Vi skal have uddannelser i verdensklasse. Målet er, at eleverne i folkeskolen bliver blandt verdens bedste til læsning, matematik og naturfag. At mindst 95 pct. af de unge får en ungdomsuddannelse i 2015. Og at mindst 50 pct. af en ungdomsårgang får en videregående uddannelse i 2015.
}
{Statsminister \Citeauthor{rasmussenStatsministerAndersFogh2005}, tale ved folketingets åbning, 24. februar 2005}
\end{epigraphs}



Undervisningsminister Ole Vig Jensen funderer for eksempel over, hviorvidt de u(ud)dannede unge kan udgøre en trussel mod demokratiet.
 Vel et årti senere lover statsminister Anders Fogh Rasmussen uddannelse “i verdensklasse”.

 Om der er fokus på, at klare sig i den internationale konkurrence eller, at få “alle” med; så er der få politikere, der kan undlade at have en holdning til uddannelse.

\subsection{Problemfelt}
\label{sec:problem}
Der ligger dog mere under overfladen, når man taler om “uddannelse”.

Uddannelsessystemet fungerer, for eksempel, de facto som en sorteringsmekanisme.
Vi kan ikke alle være historikere, læger eller sociologer - samfundet har også behov for pædagoger, slagsassistenter og murere.
Der er dog en udbredt forståelse af Danmark som et forholdsvist fladt samfund, hvor “alle” har mulighed for at blive “alt”, til trods for, at der dokumenterede forskelle i uddannelsesudfald på baggrund af socioøkonomisk udgangspunkt.\todo{kilde:  det socialt flade danmark}
Dette forstærkes (muligvis) af en form for privilegieblindhed i Danmark.  \todo{privliegieblindhed i DK. skal uddybes, og underbygges}

Dermed klinger den højt agtede sociale mobilitet også lidt hult.
Hvis uddannelsessystemet har som funktion, at sortere borgermassen i differentierede arbejdsområder, er der også elementer af, en legitimering af strukturelle forskelle \todo{kilde: legitimering af strukturelle forskelle}.


\subsection{Sortering til ungdomsuddannelserne i dag}
\label{sec:sorting}
Der er en faldende tilslutning til erhvervsuddannelserne.
I perioden 200? til 201? har andelen af unge, der søger ind på gymnasiet steget med 20(?) procent \todo{find optagelsestal}.
Andelen af elever, der optages på EUD, er faldet med ?? procent i samme periode.
Holdt sammen med, at frafaldet er meget højere på erhvervsuddannelserne, er der blandt andet bekymringer om, man kan samle “restgruppen” op. Andelen uden uddannelse ud over folkeskolen per årgang har ligget stabilt på omkring 20 procent de sidste ?? År.\todo{hvor længe har restgruppen været nogenlunde stabil?}

Qua de strukturelle uligheder der er nævnt tidligere, er der også forskelle i prestige og social agtelse i forskellige uddannelser.
Dansk Erhverv bekymrer sig fx om, hvorvidt EUD fravælges, blot fordi der går lavstatus i EUD / gymnasiet ses som “den rigtige” uddannelse\todo{kilde: Dansk erhverv EUD fravælges}.
Der tales også om, at det danske samfund er ved at blive “overuddannet”, hvilket kan være til hinder for vækst og fremgang i Danmark \autocite{simonsenLadOsGore2016}.

Hvordan man helt præcis stiller sorteringsmekanismerne i uddannelse ind, kan man (blive ved med) at skrive meget om.
Jeg er dog meget mere interesseret i, hvordan politikere taler om deres holdninger til uddannelse; gerne i et historisk perspektiv.
For at undersøge dette vil jeg foretage computeranalyser af talerne fra møderne i det danske folketing.
Min undersøgelse vil dermed omfatte historiske tendenser fra anden halvdel af det 20. århundrede til i dag, idet referaterne af disse møder er tilgængelige digitalt fra 1953.\todo{kluntet sætning}

\subsection{Problemformulering}
\label{sec:pf}
Jeg vil i dette speciale se nærmere på, hvordan politikere har italesat uddannelse i folketingssalen, ved hjælp af computeranalyser af folketingsmøder fra 1953 til i dag.
Jeg vil have ungdomsuddannelserne i fokus for min undersøgelse, herunder især den politiske diskurs omkring de erhvervsrettede ungdomsuddannelser.
Jeg vil også berøre de gymnasiale uddannelser, som kontrast og modpol til den politiske omtale af EUD.

\subsubsection{Undersøgelsesspørgsmål}
\label{sec:res-qs}

Hvilke omtaler af erhvervsuddannelserne er kendetegnende for specifikke perioder?

Er der politisk enighed eller uenighed?

Kan de politiske udtalelser omkring uddannelse gøres op efter partipolitiske linjer?

Hvem ser man som målgruppen for de erhvervsfaglige uddannelser?

Hvordan positionerer politikerne sig og andre i kraft af sine talehandlinger?

\subsection{Specialets videre opbygning}
\label{sec:structure}

Del II vil, efter en gennemgang af anden forskning i de danske erhvervsuddannelser, omhandle historiske tendenser i dansk uddannelse.
For at have en historisk kontekst for min analyse, agter jeg at gå mere i dybden hvad angår de erhvervsrettede ungdomsuddannelser, med et oprids over reformer og ændringer fra 1953 til i dag.

I del III vil jeg beskrive mine metodiske og teoretiske overvejelser.
Styrker og svagheder ved computerbaserede tekstanalyser vil blive gennemgået, og de konkrete metoder beskrevet.
Jeg vil også drøfte sociologiske ståsteder\todo{hvilken teori? er meget åben} der kan underbygge min analyse, herunder positioneringsteori, samt drøftelser omkring uddannelsens betydning for videreførelse af sociale strukturer.

\todo{Analyse/konklusioner uddybes med tiden}
I del IV findes en gennemgang af mine analyser.

Jeg afslutter specialet i del V, med en diskussion af mine resultater, og hvad betydningen kan være for videre forskning.
