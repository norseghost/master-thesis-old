\renewcommand*{\afterpartskip}{
\vfil
\begin{epigraphs}
\qitem{\itshape
  It is, I think, the political task of the social scientist who accepts the ideals of freedom and reason, to address his work to each of the other three types of men I have classified in terms of power and knowledge.
To those with power and with awareness of it, he imputes varying measures of responsibility for such structural consequences as he finds by his work to be decisively influenced by their decisions and their lack of decisions.
To those whose actions have such consequences, but who do not seem to be aware of them, he directs whatever he has found out about those consequences. He attempts to educate and then, again, he imputes responsibility.
To those who are regularly without such power and whose awareness is confined to their everyday milieux, he reveals by his work the meaning of structural trends and decisions for these milieux, the ways in which personal troubles are connected with public issues; in the course of these efforts, he states what he has found out concerning the actions of the more powerful.}{C. Wright Mills, \citetitle{millsSociologicalImagination2000}}
\qitem{\itshape
  Computation allows us to model, measure, and modify both social structures and the texture of individual experience— and to do so on bigger and smaller scales than ever before. The sociological imagination has been blown wide open. We must not forget that the new possibilities unleashed by the digital age—formal worlds and digital observatories, intelligent surveys, virtual laboratories, and machine discovery—require the socio- logical imagination to achieve their full potential.
}{James Evans og Jacob G. Foster, \citetitle{evansComputationSociologicalImagination2019}}
\end{epigraphs}
}

\part{Konklusion}

\ldots

Jeg har vist, hvordan politiske holdnigner til ungdomsuddannelser har ændret karakter siden 1953.

Der er en vekselvirkning mellem...
