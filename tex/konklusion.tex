\renewcommand*{\afterpartskip}{
\vfil
\begin{epigraphs}
\qitem{\itshape
It is, I think, the political task of the social scientist who accepts the ideals of freedom and reason,
to address his work to each of the other three types of men I have classified in terms of power and knowledge.

To those with power and with awareness of it,
he imputes varying measures of responsibility for such structural consequences as he finds by his work
to be decisively influenced by their decisions and their lack of decisions.

To those whose actions have such consequences,
but who do not seem to be aware of them, 
he directs whatever he has found out about those consequences.
He attempts to educate and then, again, he imputes responsibility.

To those who are regularly without such power and whose awareness is confined to their everyday milieux,
he reveals by his work the meaning of structural trends and decisions for these milieux,
the ways in which personal troubles are connected with public issues;
in the course of these efforts,
he states what he has found out concerning the actions of the more powerful.}{C. Wright Mills, \citetitle{millsSociologicalImagination2000}}
\qitem{\itshape
  Computation allows us to model, measure, and modify both social structures and the texture of individual experience —
  and to do so on bigger and smaller scales than ever before.
  The sociological imagination has been blown wide open.
  We must not forget that the new possibilities unleashed by the digital age —
  formal worlds and digital observatories, intelligent surveys, virtual laboratories, and machine discovery —
  require the sociological imagination to achieve their full potential.
}{ James Evans og Jacob G. Foster, \citetitle{evansComputationSociologicalImagination2019}}
\end{epigraphs}
}

\part{Konklusion}

\chapter{Store data - store indsigter?}

Jeg har vist, hvordan politiske holdninger til  uddannelse har artet sig fra 1978 til 2019.

Men 
Der er til

Der er en vekselvirkning mellem...

\section{Tekst som produkt - og som data}

Dette har været en (delvist) kvantitativ tilgang til et kvalitativt udgangspunkt.
Dette giver muligheder; primært i forhold til mængden af tekst jeg har kunnet forholde mig til.
Computeren er også i mindre grad plaget af forudindtagelser og bias end jeg
\footnote{Eller rettere; det er mig som forsker, der tilfører algoritmerne eventuelle bias.};
og har dermed muligheder for at se mønstre og tendenser der kunne forbigå\todo{er der eksempler herfor i materialet?} et sølle menneske.

Men, som nævnt i del III, er denne kvantificering af det kvalitative fordrende en homogenisering af materialet inden analyse.
Tabet af ekstremerne er et muligt tab af information; der skal tages højde for i fortolkning af data.

Den tekst jeg har kværnet igennem min algoritmiske kødhakker har, inden det blev transkriberet, startet som politisk tale; med helt særlige institutionelle rammer.
Det er, for eksempel, et specielt taleprodukt i forhold til hvem den henvender sig til.
Når man taler om “uddannelse for alle” eller “uddannelse i verdensklasse”, for eksempel,
henvender man sig samtidigt til andre politikere; sine vælgere; og offentligheden i almindelighed.
Her kan ekstreme observationer være vigtige;
om end jeg står ved mine observationer.


\section{Værdien af en beskrivende sociologi}

Jeg refererede i min indleding til \citeauthor{savageContemporarySociologyChallenge2009} (\citeyear{savageContemporarySociologyChallenge2009}, der går fuldt ind for en deskriptiv sociologi.
Disse bemærkninger er ikke gået ubemærkede hen; og \citeauthor{ganeDescriptiveTurn2020} advarer mod at tabe dybden i den sociologiske forskning --- forklaringer, analyser, kausalitetsspørgsmål --- i en deskriptiv iver.

Jeg har tydeligt situeret dette speciale i en \textit{beskrivende} sociologisk tradition.
Dette — meget bevidste — valg giver mig en frihed til,
at se på \textit{bredden} i mit empiriske grundlag.
Jeg kan påpege tendenser i politiske ståsteder;
jeg kan udforske trends i holdning og tone blandt politikere;
jeg kan udforske klynger af emner der optræder sammen;
jeg kan se hvordan de politisk centrale emner ændrer karakter.

Dette ser jeg som et vigtigt og værdifuldt produkt i sig selv.
Ikke mindst lige vigtigt er mulighederne et velbeskrevet felt giver for videre analyse.
Men, som \citeauthor{millsSociologicalImagination2000} understreger, er beskrivelser i sig selv ikke nok.
Der er en politisk opgave, mener han, at oplyse både magthavere og de magtesløse om, hvad kollektive beslutninger kan betyde for den enkelte \autocite[s. ]{millsSociologicalImagination2000}.
Men beskrivelsen er ikke nok for \citeauthor{millsSociologicalImagination2000}.
Han understreger også, at det er agtpåliggende at få blotlagt de kausale forbindelser mellem personligt miljø og sociale strukturer (\citeyear[s. 130]{millsSociologicalImagination2000}).
En beskrivelse kan ikke stå alene — der skal også gerne være en forklaring på et tidspunkt.

Her er der rigelige muligheder for at fortsætte i \textit{dybden}, og uddybe forklaringerne.

\section{Beskrive - på baggrund af hvad?}

På et meta-niveau er det nyttigt for videnskaben,
at facilitere og forhåndsbehandle data,
og dele dette arbejde med verden.
Jeg har selv nydt godt af andres arbejde i at gøre grovarbejdet i,
at gøre Folketingstidendes datasæt tilgængeligt for videre bearbejdning.
Jeg har dog, som bemærket flere steder i mit arbejde, 
også stødt på flere mangler og gråsoner i løbet af arbejdet med dette speciale.
Jeg vil meget gerne selv give tilbage til verden, og bidrage til, at imødekomme dette.
Godt forskningsarbejde er nemmere med et godt datasæt at arbejde med.

Mine data og min kode er også tilgængelige for offentligheden \autocite{andersenNorseghostMasterthesis2020}; til frit skue og inspiration.

\chapter{Fokus for videre forskning}

Det vil være spændende, at fortsætte computeranalyserne med andre modeller for at udlede emner.
LDA er en af de velkendte tilgange til \textit{topic modeling}, men slet ikke det eneste mulige værktøj.
At arbejde med en anden tokenisering vil også være værd at udforske — 
trigrams og skip-grams, for eksempel, der blev udelukket i denne omgang af tidshensyn.
Og man kan også arbejde i højere grad med analyser på de enkelte sætninger.

Det vil være spændende, at fortsætte arbejdet med andre modeller for at udlede emner - LDA er en af de velkendte tilgange, men slet ikke den eneste mulige tilgang.
At arbejde med en anden tokenisering vil også være værdt at udforske — trigrams og skip-grams blev udelukket i denne omgang af tidshensyn, for eksempel. Og man kan også arbejde i højere grad med analyser på sætningsniveau.
Det var også muligt at antyde emner, der kunne have interesse for andre forskningsfelter.
For eksempel emnet ???? ???? ?, adsdasasdfa., .afdsdfa; der ser ud til at omhandle sundhedsvæsnet; eller.
