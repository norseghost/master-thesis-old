\renewcommand*{\afterpartskip}{
\vfil
\begin{epigraphs}
\qitem{\itshape
“Jo, hvis dét skal kaldes Fakta, saa benægter a Fakta!”
}{folketingsmedlem Søren Kjær, i debat med Carl Steen Andersen Bille}
\qitem{\itshape
  Politik skal ikke videnskabeliggøres. Der findes ikke noget facit i politik – kun følelser og holdninger. Begreber som sandt og falsk eller godt og ondt har ganske enkelt ikke hjemme i det politiske rum. 
}{Peter Skaarup, i et ugebrev for Dansk Folkeparti, \citeyear{skaarupPolitikErForst2017}}
\end{epigraphs}}

\part{Analyse}\label{part:analysis}

\chapter{At beskrive den sociale virkelighed}

Denne opgave tager udgangspunkt i, at der antages at være en intersubjektiv social virkelighed, og det er muligt at undersøge denne.
Jeg læner mig videre op af et socialkonstruktionistisk verdensbillede; hvor denne sociale verden er i kontinuerlig tilblivelse i en kollektiv samskabelsesproces \autocite{gergenSocialkonstruktionismeOgUddannelse}.
Et centralt element i denne samskabelse af en fælles social virkelighed er \textit{sam}talen \autocite[s. 15f]{gergenSocialkonstruktionismeOgUddannelse}.
Ord skaber og former virkeligheden.
Ikke dermed sagt, at der altid er enighed omkring konturene af denne samskabte sociale virkelighed — heller ikke i politik.
Dette understreges, måske lidt kækt, af Søren Kjær på titelbladet for denne del af specialet.
Ikke dermed forstået, at man blankt skal afvise påstande man er uenig i; men at man gerne skal have belæg for hvad, man fremlægger som “fakta”.
Peter Skaarup argumenterer derimod med et moralistisk udgangspunkt.
Man ved hvad der er “rigtigt”; og har ikke behov for beviser, for at det rigtige også er “sandt”.
Der er et spil der skal spilles; og præmien er en vis form for kontrol over, hvad der fremstår som “rigtigt”.
\footnote{Spil-metaforen, kort drøftet i del II, kan også overføres til en socialkonstruktionistisk ramme:
spillets regler udgør et bagtæppe for deltagernes verdensbillede.
Dette fordrer specifikke sproglige ritualer og giver specifikke ord særlige betydninger \autocite[s. 24f]{gergenSocialkonstruktionismeOgUddannelse}.}

Jeg vil i det følgende bestræbe mig på, at både præsentere og underbygge mine fakta, i kraft af at ikke have en brik med i det politiske spil.
Dette gør jeg med udgangspunkt i “trappen” fra del III.
Undervejs vil jeg belyse mine undersøgelsesspørgsmål, efterhånden som det falder naturligt.
Jeg vil også illustrere de særegenheder jeg måtte finde i mit datasæt undervejs; da dette har stor betydning for, hvordan jeg griber både forarbejdet og selve analysen an.

\chapter{Indskrænkning af dokumenter til undersøgelse}

Jeg begynder med, at udarbejde et tf-idf objekt for hver analyseperiode.
En analyseperiode udgør således et korpus, hvor teksten er blevet forhåndsbehandlet og klargjort.
For at øge analyseværdien har jeg udeladt ord der falder under gennemsnitsværdien for \textit{tf-idf} for dermed at sortere de aller mes hyppigt forekommende ord fra.
For at undgå, at de aller mest sjældne ord forvrænger analysen, udelukker jeg herefter ord der har en \textit{tf-idf} i de yderste to promille.

Derefter foretager jeg en sammenligning af forskellige modeller for beregning af det optimale emner for videre analyse.

\begin{figure}
\input{../fig/models_bigrams_no_stopwords_5to125by10.tex}
\caption{Beregning af optimalt antal emner for videre analyse; bigrams og ingen stopord.}
\label{fig:modelsFull}
\end{figure}

Figur ~\ref{fig:modelsFull} (side ~\pageref{fig:modelsFull}) viser en sammenligning af 4 forskellige modeller for udvælgelse af et "optimalt" antal emner for en LDA-baseret topic model.
Ved at se på kurvene fra disse sammenligninger, konvergerer kurvene omkring de 25-65 emner, inden de bevæger sig opad igen\footnote{Med undtagelse af \autocite{deveaudAccurateEffectiveLatent2014}; der ikke ser ud til at være en særlig hjælpsom algoritme i denne sammenhæng. Muligvis en konsekvens af mit noget specielle datasæt.}.
Jeg går videre med 45 emner, da det er her; der er flest perioder med en “bund” i sine kurve.
Et nærmere blik på kurverne giver også nogle indsigter.
Perioden \textit{2001-14} flader ud helt op til ca 75 emner, i lighed med “kontrolperioden”\footnote{
De emner (og de tilhørende dokumenter) vil dog ikke være de samme som begreberne for de individuelle perioder lagt sammen; da modellen laver beregningen på bagrund af alle dokumenterne på en gang.
Jævnfør “sækken med ord” fra min metodediskussion, er der nu langt flere ord i sækken; og de skal fordeles over flere dokumenter.}
\textit{all}.
Dette er formodentlig et produkt af, at disse grupperinger er på top i antal dokumenter at gennemgå; som funktion af deres forholdsvis lange tidsperioder.
Dermed er jeg igen nået til en subjektiv vurdering:
Skulle man give hver periode forskelligt antal emner?
Jeg vælger at ensrette antal emner. 
Dette både for sammenligningen mellem perioders skyld;
men også af tidshensyn.

En LDA beregning giver ikke navngivne emner, som sådan.
I kraft af at være en ikke-superviseret algoritme, bliver dokumenterne fordelt ud over det anviste antal emner\footnote{LDA tildeler dokumenterne en score $\gamma$ mellem 0 og 1 for deres bidrag til et givet emne. Emner er distributioner over begreber; hvor hvert begreb tildeles en score $\beta$ mellem 0 og 1 for at tilhøre et givet emne. \\ \\ Hvert emne $\phi k$ er en multinominel distribution over ordforrådet $W$, og hvert dokument $\theta  d$er en multinominel distribution over $K$ emner. Dermed er sandsynligheden for at et givet dokument ($d$) bidrager til emnet $k$ $\theta _{d,k}$; og sandsynligheden for at ord $w$ tilhører et emne $\phi _{k,w}$. De multinominale distributioner for emnet genereres af en konjugat Dirichlet prior((FIXME: dansk begreb?)) $\overrightarrow{\beta}$, og distributionerne for dokumenter på baggrund af en Dirchlet prior $\overrightarrow{\alpha}$. Deraf navnet \textit{“Latent Dirichlet Allocation”} \autocite[s.65f]{deveaudAccurateEffectiveLatent2014}} , i det omfang (algoritmen mener) de ligner hinanden.
Hvorvidt der er grupperinger der giver mening for mennesker i en social kontekst, og hvilken mening de eventuelt skulle have, er op til forskeren at vurdere.
Som eksempel præsenteres et udvalg af emnerne genereret for perioden 1990 til 2000 i figur ~\ref{fig:termsFull} (side ~\pageref{fig:termsFull}).
Emnerne er repræsenteret af de 15 hyppigste termer pr emne, rangeret efter begrebernes vægtning for det pågældende emne.

\begin{figure}
\begin{adjustwidth}{-1in}{-1in}
  \input{../fig/terms_bigrams_no_stopwords.tex}
\end{adjustwidth}
\caption{Oversigt over udvalgte emner for perioden 1990-2000, med tilhørende begreber.}
\label{fig:termsFull}
\end{figure}

\textbf{Emne 4} er et eksempel på, de mange emner der genereres omkring “administrivia”.
Folketingsmøder har et særligt, formelt sprog, og dette genspejles i rigtigt mange emner, der omhandler denne slags formalia\footnote{En grov optælling tilsiger ca. 10-12 af denne slags emner for hver periode; med “kontrolperioden” op over 20.}

Der er dog alligevel meget klart, at nogle emner har (pædagogisk) sociologisk relevans.
\textbf{Emne 5} omhandler tilsyneladende finanspolitik\footnote{Et lignende emne, med mange af de samme begreber, er at finde i alle perioder}.
Man kan ane den førnævnte aktive socialpolitik i \textbf{emne 6}, og krigerne i Irak og Afghanistan i \textbf{emne 13}.
Men mest interessant for mit vedkommende er \textbf{emne 20} og \textbf{emne 23}, der er repræsenteret af termerne \textit{“ung menneske”}, \textit{“videregående uddannelse”}, \textit{“i folkeskole”} og \textit{“dansk sprog”} blandt andre.
Forhenværende børne- og undervisningsminister Christine Antorini gør også en optræden.
Min udtalelse fra indledningen --- det er få politikere der kan undgå at tale om uddannelse — er også underbygget; idet jeg kan finde emner der indeholder beslægtede begreber for hver analyseperiode.

Jeg vil gerne forsøge at øge mit forhold mellem signal og støj.
Vil man kunne øge informationstætheden ved at udelukke de mange formalia-relaterede begreber?
Dette er et oplagt scenarie til brug af \textit{stopord}.
Jeg bruger en forhåndsgenereret samling af generelle stopord \autocite{stopwords-isoStopwordsISO2020};
i tillæg til en liste domænespecifikke stopord\footnote{
for den konkrete stopordsliste se \texttt{lib/stopwords.txt} i specialets Gitub-repositorie \autocite{andersenNorseghostMasterthesis2020}.}
jeg selv har udfærdiget.
Dette viser sig dog at være en blindgyde.
Kurverne er nu uden det pæne “buk”\footnote{Igen, på nær \autocite{deveaudAccurateEffectiveLatent2014}}, og svinder ned mod 0; uden der er en opadstigende kurve.
Et uddrag af begreber, med antal emner sat til 45, giver tilsvarende utilfredsstillende resultater.
At sortere stopordene fra har tilsyneladende medført for stort informationstab til at der er analyseværdi tilbage\footnote{
Det samme gjorde sig gældende ved andre tiltag for at reducere kompleksiteten i mit materiale —
det være sig andre øvre eller nedre terskelværdier for filtrering i \textit{tf-idf};
eller en reduktion af sjældne terner i de genererede \textit{Document-Term Matrices}.}.
Muligvis ville det være muligt at tilpasse listen af stopord, men dette er blevet udeladt af tidshensyn\footnote{Det tager flere dage til over en uge at sammenligne modellerne}.

Jeg foretager dermed de videre analyser uden stopordslister.
Men jeg kommer igen til et vurderingsspørgsmål:
Det er nogle perioder, der har to emner der kan relatere til uddannelse bredt forstået; og andre der kun har en.
Jeg vurderer, at mere data er godt i denne sammenhæng; og tager alle relevante emner med for hver periode.

Men inden jeg dykker ned i uddannelse specifikt, vil jeg gerne se på, hvordan mit datasæt ser ud i forhold til sig selv.

\section{Talernes indbyrdes relationer}
Som tidligere nævnt har hvert datasæt sine særegenheder.
Jeg formodede i mit metodeafsnit, at der var en forholdsvis stor ensartethed i min samling af taler.
Min hypotese er, at dette skyldes den noget kunstfærdige situation en tale i Folketinget befinder sig i.
Talernes kontekst, lejret i det politiske spil, bringer en række rammer for talerne; der i sidste ende formodentlig får en tale til at ligne mange andre.
Dette er illustreret ved figur ~\ref{fig:dendro_all};
hvor emnernes indbyrdes relationer for perioden \textit{2001-14} fremgår\footnote{
Graferne for de andre perioder kan ses på \href{martinandreasandersen.com/projects/au/}{min blog} \autocite{andersenSelectedAssignmentsAarhus}}.
For at bringe denne diskussion tilbage til en sociologisk relevans, kan også visualisere hvilke emner, computeren mener er nært beslægtede\footnote{Dette er en proces kaldet \textit{hierarchical clustering}. Alle emner tildeles en klynge; hvorpå de klynger der er tættest på hinanden slås sammen. Jeg benytter Ward's algoritme til dette, hvor klynger med mindst indbyrdes variation grupperes \autocite{wardHierarchicalGroupingOptimize1963}.} med emnerne omkring uddannelse.\footnote{Jeg har indikeret de emner, jeg trak ud i forrige figur med farve.}
Vi kan fange vores emner omkring uddannelse, repræsenteret af begreberne \textit{“den ung”} og \textit{“den fri”}.

Også i grendiagrammet er de tæt beslægtede.
De ser videre ud til, at alle være en del af en stor klynge, der omhandler det almindelige politiske virke.
Alle mine eksempelemner figurerer i denne klynge.
Der er videre ikke stor forskel på bladenes højde, i kraft af et forholdsvist homogent datasæt. 

\begin{figure}
  \input{../fig/cluster_bigrams_no_stopwords_2001-14_k45.tex}
  \caption{LDA-modellens emner for perioden 2014-20, efter indbyrdes relationer. Emner repræsenteres med deres mest fremtrædende begreb.}
\label{fig:dendro_all}
\end{figure}

\hrule

Næste trin i min analyse bliver nu, at se på indholdet i mine udvalgte emner.

Jeg fortsætter, som nævnt, uden stopord; og undersøger de forskellige korpora hver især, og trækker dem ud, der ser ud til at omhandle uddannelse\footnote{Den interaktive visualisation jeg benytter til at udforske disse LDA-modeller kan (også) findes på \href{martinandreasandersen.com/projects/au/}{min blog} \autocite{andersenSelectedAssignmentsAarhus}}.

\chapter{Analyse af de udvalgte dokumenter}
Jeg har nu en samling emner, der ser ud til, at kunne omhandle uddannelse.
De dokumenter der har højest sandsynlighed for at tilhøre disse emner skal nu trækkes ud til videre analyse, idet en LDA-model, som nævnt, vil tildele alle dokumenter en sandsynlighed for at tilhøre et bestemt emne.
Jeg vælger de taler, der falder i de øverste 2\%  af sandsynlighed for, at tilhøre mine udvalgte emner.
Dette giver en minimumsværdi for sandsynligheden for at dokumenterne tilhører mine udvalgte emner ($\gamma$) på omkring $~0.1\ldots-0.2$.
Dette er givetvis ikke særlig højt; men en lavere andel gav rigtigt få dokumenter at arbejde med\footnote{Jeg prøvede også med en minimums\textit{værdi} for $\gamma$; men denne skulle være meget lav før jeg fik en nævneværdig samling dokumenter retur. Derudover var der nogen variation i højeste værdi for $\gamma$ i de forskellige perioder; med tilsvarende variation i antal dokumenter for en given minimumsværdi.}.

Med mine dokumenter trukket ud; er der nu tid til en videre analyse.

Først vil jeg se, hvilke emner der er gennemgående i dette subset af dokumenter.
Derefter vil jeg undersøge, hvorvidt der er emner at trække ud indenfor disse dokumenter; da jeg ikke kun er interesseret i uddannelse generelt, men har en specifik interesse i ungdomsuddannelserne.
Jeg vil både se nærmere på dette eventuelt forskningsmæssigt interessant subset af data; og også holde øje med, om der er dokumenter der tydeligt ikke omhandler uddannelse inden jeg fortsætter min analyse.

Jeg vil derefter forsøge at at afdække partipolitiske positioner over dette subset.

\section{Emner indenfor uddannelse}

Efter en ny beregning af et optimalt antal emner, fortsætter jeg med... som gunstigt antal emner for undersøgelse af mit (formodede) uddannelses-subset.
Som det fremgår af grendiagrammet i ~\ref{fig:dendro_edu}, er der dog blevet sværere, at differentiere mellem et optimalt antal emner...
Jeg ser nogle emner gå igen — meget administrivia, formalia og så videre.
Men nu kan jeg også i højere grad differentiere mellem underkategorier.
Der er taler omkring ????? ?????? ????? ??  - som til sammen bliver noget med “erhvervsuddannelser”
Og taler der handler om ?????? ??????? ??????? ????? ?????? - der kan smales under “gymnasier”

Et nyt grendiagram (figur ~\ref{fig:dendro_edu} viser også, at disse begreber er beslægtede.
Grendiagrammet er mere homogent nu, hvilket....


\begin{figure}
 \input{../fig/cluster_edu_test.tex}
\caption{Grendiagram over emner indenfor uddannelse, med indbyrdes relationer fremhævet i farver.}
\label{fig:dendro_edu}
\end{figure}


De fleste emner i dette nye subset ser ud til at omhandle uddannelse af en art.

\begin{figure}
\begin{adjustwidth}{-1in}{-1in}
  \input{../fig/terms_edu_test.tex}
\end{adjustwidth}
\caption{Oversigt over udvalgte emner for perioden 1990-2000, med tilhørende begreber.}
\label{fig:termsEdu}
\end{figure}

\section{Partipolitiske positioner}

Ordoptellingen ovenfor er noget begrænset i forhold til at afdække partipolitiske positioner.
Man kunne, for eksempel, se på hvilke begreber der var sigende for den enkelte periode; og dermed muligvis se en generel trend over tid.
Men der er udarbejdet konkrete algoritmer for at undersøge dette.
Jeg vil benytte mig af den før omtalte \textit{Wordfish}-algoritme \autocite{slapinScalingModelEstimating2008}.

Denne algoritme tager for sig en samling dokumenter; hvor hvert dokument har en (af forskeren, på forhånd etableret) partipolitisk tilhørighed; og også ideelt set omhandler nogenlunde det samme emne.
\citeauthor{slapinScalingModelEstimating2008} demonstrerer deres analyse på partiprogrammer - først i deres helhed; derefter delt op i politiske interesseområder\footnote{\textit{Wordfish}(et ordspil over ord og den franske betydning af \textit{poisson}) modellerer ordfrekvenser mod en Poisson-distribution; og tildeler en en-dimensionel værdi på baggrund af dette. Dette er også medvirkende til, at algoritmen har det bedst med dokumenter der falder indenfor samme emne; da man ellers risikerer en meget stor spredning i ordfrekvenser dokumenterne i mellem.}.
I mit tilfælde tager jeg for mig mine dokumenter der skal analyseres videre; og tildeler dem en politisk fløj på baggrund af politisk parti\footnote{Der er en mangel i mit datasæts metadata; hvor der ikke forefindes partitilhørighed på talere som “statministeren”; “formanden” og så videre. Disse er tildelt “ikke angivet” som blok.}.
Derudover kategoriseres Grønland og Færøerne for sig.
x
\begin{table}
\caption{Fordeling af partier i politiske blokke}
\label{tab:party2bloc}
\begin{adjustwidth}{-0.8in}{-0.8in}
\begin{tabular}{@{}llll@{}}
\multicolumn{1}{c}{\textbf{Blå blok}} & \multicolumn{1}{c}{\textbf{Rød blok}} & \multicolumn{1}{c}{\textbf{Centrum}} & \multicolumn{1}{c}{\textbf{Grønland/Færøerne}} \\ \midrule
Dansk Folkeparti        & Alternativet                  & Uden for partierne    & Javnaðarflokkurin                           \\
Venstre                 & Enhedslisten                  & Radikale Venstre      & Tjóðveldisflokkurin                         \\
Konservative Folkeparti & Socialdemokraterne            & Kristeligt Folkeparti & Fólkaflokkurin                              \\
Fremskridtspartiet      & Socialistisk Folkeparti       & Centrum-Demokraterne  & Sambandsflokkurin                           \\
De Uafhængige           & Danmarks Kommunistiske Parti  & Ny Alliance           & Atássut                                     \\
Nye Borgerlige          & Venstresocialisterne          & Liberal Alliance      & Siumut                                      \\
                        & Fælles Kurs                   & Retsforbundet         & Inuit Ataqatigiit                            \\
                        & & Liberalt Centrum            &                                                \\ \bottomrule
\end{tabular}
\end{adjustwidth}
\end{table}


\textit{Wordfish}-algoritmen beregner positioner i én dimension, fortolket som en venstre-højre akse ud fra dokumenternes ordfordeling sig i mellem.
Dokumenternes position holdes uafhængig af hinanden; således at et partis position i periode $t$ ikke påvirkes af positionen i periode $t-1$.
Forbliver partiernes positioner forholdsvis uændrede, har partierne brugt lignende ord over tid.
Er der derimod ændringer i hvilke ord der bruges over tid, vil dette vise sig i en ændret position over tid.

Jeg udarbejder et nyt korpus, hvor hvert korpus udgør talerne for hver blok, der slås sammen til ét dokument pr periode\footnote{Dog er partier og/eller blokke der ikke fremgår af figurer og tabeller også udeladt fra analyserne. Da worfish-algoritmen sammenligner alle de medtagne dokumenter i sin vægtning, vil det forvride den relative præsentation af partiernes indbyrdes forhold}.
Ud fra dette tæller jeg op antallet tokens per dokument, og laver beregninger ud fra dette.
Til sammenligning laver jeg en lignende beregning, fordelt over partiernes positionering over regeringsperioder.

\begin{figure}
  \input{../fig/wordfish_blocs_periods.tex}
\caption{Politiske blokkes politiske positioner over analyseperioderne}
\label{fig:fish_blocXperiod}
\end{figure}Her ser vi, hvordan blokkene har ændret position i hver analyseperiode.
Der er nogen bevægelse i de forskellige blokke, men de relative positioner er forholdsvis stabile......

\begin{figure}
  \input{../fig/wordfish_party_by_period.tex}
\caption{Politiske partiers politiske positioner over analyseperioderne}
\label{fig:fish_partyXperiod}
\end{figure}

Når man ser på de enkelte partier, kan man se hvordan de bevæger sig rundt...

\begin{figure}
\begin{adjustwidth}{-8em}{-8em}
  \input{../fig/wordfish_parties_governments.tex}
\end{adjustwidth}
\caption{Politiske partiers politiske positioner over regeringsperioder}
\label{fig:fish_partyXgovt}
\end{figure}
Nu er vi mere i detaljen; både hvad angår tidsinddeling og perioder. Nu kan vi se en større spredning i de forskellige partiers holdninger.

\subsection{Hvilke begreber er kendetegnende for de politiske fløje over tid?}

Man kan også udlede, hvilke ord giver anledning til hvilke positioner.
Dette gøres ved at undersøge de værdier, \textit{Wordfish}-algoritmen har tildelt de enkelte begreber.

\begin{figure}
\begin{adjustwidth}{-8em}{-8em}
  \input{../fig/coef_edu_test.tex}
\end{adjustwidth}
\caption{Ordfordeling og vægtning for de politiske blokke. Figuren udelukker de grønlandske og færøske partier.}
\label{fig:coef_partyXperiod}
\end{figure}

Ved at se på figur ~\ref{fig:coef_blocXperiod}\footnote{Den noget karakteristiske form kommer af, at politisk neutrale ord ikke får tildelt en positiv eller negativ vægtning; og vil ikke trække et parti eller en blok i en bestemt retning. Ord der bliver brugt ofte af alle parter, med andre ord, er i udgangspunktet politisk neutrale \autocite[s. 709]{slapinScalingModelEstimating2008}}, fremgår der, at ord som ?????, ??????, ????? indikerer en venstrerettet position; og ord som ???? ???? ?????? ?????? angiver en højreposition.
Ordene ???, ???,, ???, der har en vægtning nær 0, er mere værdineutrale i denne sammenhæng.

Figur ~\ref{fig:coef_blocXperiod} viser ordfordeling og vægtning for hele mit datagrundlag.
Man kan også udlede samme data for de enkelte analyseperioder.
Ordfordelingen for hver periode kan afleses i tabel ~\ref{tab:lrterms}, hvor de centrale begreber der giver venstre, højre, og neutral vægtning for hver periode er fremhævet.

\begin{table}
\caption{Oversigt over begrebsvægtning efter en \textit{Wordfish} beregning af talerne i analyseperioderne.}
\label{tab:lrterms}
\begin{adjustwidth}{-9em}{-9em}
\begin{tabular}{lp{2in}p{2in}p{2in}}
\input{../fig/table_coef_terms_edu_test.tex}
\end{tabular}
\end{adjustwidth}
\end{table}

\subsection{Hvordan taler politkere om uddannelse?}
Ovenfor har jeg givet en visualisering af de politiske partiers relative positioner omkring uddannelse.
Der er ikke de store overraskelser; om end det er påfaldende at...

Dette siger dog ikke noget om \textit{tonen} i talerne.
Ved hjælp af biblioteket \texttt{SENTIDA} udarbejder jeg en holdningsanalyse, hvor jeg tildeler talerne en holdningsværdi; både for talen i sin helhed og for de individuelle sætninger talene er bygget op af.
Er talen/sætnignen overvejende positiv; får den en positiv værdi, og er den negativ, giver det en negativ værdi\footnote{Der udarbejdes to værdier, for at være præcis. En for gennemsnitsværdien af teksten, og en for totalværdien af teksten. Jeg har derudover trukket minimums- og maksimumsværdier for talernes sætninger; samt en gennemsnits- og medianværdi for sætningernes holdningsscore}.
Dermed kan man undersøge, hvilke grupperinger i Folketinget taler overvejende positivt eller negativt.
Er der påfaldende mønstre; kan man videre trække de enkelte taler og sætninger ud.

Figur ~\ref{fig:sentposneg} viser partiers\footnote{Jeg holder mig til det udvalg af partier jeg har benyttet mig af tidligere} spredning mellem lave og høje totalværdier for sætninger\todo{analyse på blokniveau?}, fordelt over mine analyseperioder.

Det fremgår, at...

\begin{figure}
  \input{../fig/sent_minmaxedu_test.tex}
\caption{Sprednig mellem talernes højeste og laveste holdningsværdi, på sætningsniveau.}
\label{fig:sentminmax}
\end{figure}

Ser man på gennemsnitsværdierne (figur ~\ref{fig:sentminmax})  for de enkelte sætninger, ser det knapt så dramatisk ud.

\begin{figure}
  \input{../fig/sent_posnegedu_test.tex}
\caption{Sprednig mellem talernes gennemsnitlige holdningsværdi, på sætningsniveau.}
\label{fig:sentposneg}
\end{figure}

\begin{figure}
\begin{adjustwidth}{-1.6in}{-1.5in}
  \includegraphics{../fig/sent_spread_time_edu_test.pdf}
\end{adjustwidth}
\caption{Oversigt over udvalgte emner for perioden 1990-2000, med tilhørende begreber.}
\label{fig:sentspread}
\end{figure}


\section{Omtale af uddannelsespolitiske målgrupper}

\section{Ord skaber og reificerer samfundets strukturer}

