\part{Analyse}\label{part:analysis}
\begin{epigraphs}
\qitem{\itshape
“Jo, hvis dét skal kaldes Fakta, saa benægter a Fakta!”
}{folketingsmedlem Søren Kjær, i debat med Carl Steen Andersen Bille}
\qitem{\itshape
  et citat mere — noget med ekspertlede og 
}{klog person}
\end{epigraphs}
\chapter{At beskrive den sociale virkelighed}


Denne opgave tager udgangspunkt i, at der antages at være en intersubjektiv social virkelighed, og det er muligt at undersøge denne.
Jeg læner mig videre op af et socialkonstruktionistisk verdensbillede; hvor denne sociale verden er i kontinuerlig tilblivelse i en kollektiv samskabelsesproces \todo{kilde socialkonstruktionisme}.
Ikke dermed sagt, at der altid er enighed omkring konturene af denne samskabte sociale virkelighed — heller ikke i politik.
Dette understreges, måske lidt kækt, af Søren Kjær på titelbladet for denne del af specialet.
Der er et spil der skal spilles; og præmien er en vis form for indflydelse over, hvordan den sociale verden skal fortolkes og forståes.
Citatet fra \todo{hvem siger dette} understreger, styres politikere i høj grad af deres holdninger til den sociale verden når de forholder sig til den — også når det gælder uddannelse.

Jeg vil nu præsentere resultaterne for mine undersøgelser, samt redegøre for de blindgyder og nulresultater jeg måtte løbe ind i undervejs.
I løbet af min analyse, håber jeg at kunne få øje på noget af det \textit{subjektive} i den intersubjektive samskabelse af den sociale virkelighed.

Jeg vil arbejde mig igennem “trappen” fra del III; og undervejs forøge at belyse mine undersøgelsesspørgsmål efterhånden som der dukker noget op; så at sige.\todo{fæl sætning}

\chapter{Indskrænkning af dokumenter til undersøgelse}

Jeg begynder med, at udarbejde et tf-idf objekt pr korpus; der er blevet forhåndsbehandlet og klargjort - dog uden stopord.
Dette filtreres ved, at udelukke ord der falder under gennemsnitsværdien;\todo{inkludere kodeeksempler?} og ved derefter at udelukke ord der er i de øverste to promille af forekomst.

Derefter foretager jeg en sammenligning af forskellige modeller for beregning af det optimale emner for videre analyse:

\begin{figure}[H]
\centering
\input{../fig/models.tex}
\caption{Differentiering mellem emner for forskellige antal emner; bigrams og ingen stopord. }
\end{figure}

Ved at se på kurvene fra disse sammenligninger, begynder kurvene at flade ud omkring de 30-50 emner.
Der er en meget grund kurve inden den begynder at bevæge sig opad igen.
Vælger emner for undersøgelse ud fra “albueleds-princippet” - der hvor kurven begynder at bukke sig brat.
I dette eksempel er det omkring 35 emner.
Illustreret af perioden 1990 til 2000 giver dette følgende emner, repræsenteret af de 15 hyppigste termer pr emne:

\begin{figure}[H]
  \small
\centering
\input{../fig/1990-01-topicnumbers.tex}
\caption{oversigt over hyppigst forekommende ord pr emne}
\end{figure}


Der er en gruppering (emne ??) der ser ud til at omhandle uddannelse.
Det viser sig, at de manglende stopord gav mange emner omhandlende administrivia. For at se om jeg kan øge informationstætheden forsøger jeg en ny runde; denne gang med en række generelle plus nogle domænespecifikke stopord\footnote{for den konkrete stopordsliste se \texttt{lib/stopwords.txt} i specialets GitHub-repositorie.}.

Dette giver følgende spredning:

\begin{figure}[H]
\centering
\input{../fig/models.tex}
\caption{Differentiering mellem emner for forskellige antal emner; bigrams og stopord. }
\end{figure}

Her kan man se, at det “optimale” antal emner for denne behandling af teksten ligger mellem ???.

Ved udtræk af modellen for ??? emner, får jeg emner repræsenteret af følgende termer:

\chapter{Analyse af de udvalgte dokumenter}

Jeg har nu en samling dokumenter, der ser ud til, at kunne omhandle uddannelse.
Disse trækkes ud til videre analyse.
Først vil jeg se, hvilke termer der er gennemgående i dokumenterne, med en ordoptælling.
Derefter vil jeg undersøge, hvorvidt der er emner at trække ud indenfor disse dokumenter.
Jeg vil derefter køre en analyse, der har til sigte at afdække partipolitiske positioner over dette subset.

\section{Gennemgående termer for hver periode}

\begin{figure}
  \caption{De 20 mest omtalte termer for de udvalgte dokumenter, fordelt over de definerede analyseperioder.} 
\end{figure}

\subsection{Hvilke begreber er kendetegnende for de politiske fløje}
\todo{hvordan dele op i fløje}

\section{Emner indenfor uddannelse}

\section{Partipolitiske positioner}

Ordoptellingen ovenfor er en noget simplistisk tilgang til at afdække partipolitiske positioner.
Der er udarbejdet konkrete algoritmer for at undersøge dette.
Jeg vil benytte mig af den før omtalte \texttt{wordfish}-algoritme \autocite{slapinScalingModelEstimating2008}.

