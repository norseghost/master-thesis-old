\part{Analyse}\label{part:analysis}
\todo{find smarte citater - noget med politikeres afslappede holdninger til fakta? eller noget med erkendelse og ny viden?}

\chapter{At beskrive den sociale virkelighed}

I denne del af specialet vil jeg præsentere resultaterne for mine undersøgelser, samt redegøre for de blindgyder og nulresultater jeg måtte løbe ind i undervejs.

\chapter{indskrænkning af emner til undersøgelse}

Jeg begynder med, at udarbejde et tf-idf objekt pr korpus; der er blevet forhåndsbehandlet og klargjort - dog uden stopord.
Dette filtreres ved, at udelukke ord der falder under gennemsnitsværdien;\todo{inkludere kodeeksempler?} og ved derefter at udelukke ord der er i de øverste to promille af forekomst.

Derefter foretager jeg en sammenligning af forskellige modeller for beregning af det optimale emner for videre analyse:

\begin{figure}[H]
\centering
\input{../fig/models.tex}
\caption{Differentiering mellem emner for forskellige antal emner; bigrams og ingen stopord. }
\end{figure}

Ved at se på kurvene fra disse sammenligninger, begynder kurvene at flade ud omkring de 30-50 emner.
Der er en meget grund kurve inden den begynder at bevæge sig opad igen.
Vælger emner for undersøgelse ud fra “albueleds-princippet” - der hvor kurven begynder at bukke sig brat.
I dette eksempel er det omkring $K = 35$.
Illustreret af perioden 1990 til 2000 giver dette følgende emner, repræsenteret af de 15 hyppigste termer pr emne:

\begin{figure}[H]
\centering
\input{../fig/1990-01-topicnumbers.tex}
\caption{oversigt over hyppigst forekommende ord pr emne}
\end{figure}


Der er en gruppering (emne ??) der ser ud til at omhandle uddannelse.
Det viser sig, at de manglende stopord gav mange emner omhandlende administrivia. For at øge informationstætheden forsøger jeg en ny runde; denne gang med en række generelle plus nogle domænespecifikke stopord\footnote{for den konkrete stopordsliste se \texttt{lib/stopwords.txt} i specialets GitHub-repositorie.}.

Dette giver følgende spredning:

\begin{figure}[H]
\centering
\input{../fig/models.tex}
\caption{Differentiering mellem emner for forskellige antal emner; bigrams og stopord. }
\end{figure}


