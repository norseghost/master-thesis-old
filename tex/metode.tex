\section{metode}

\subsection{dataindsamling og analyse}

gennemgang af folketingstaler fra 1953--> nu

data fra scraping af folketingstidende.dk;
fejlkilder: ocr/tekstkonverteringsfejl; mis-attributering;

databearbejdning - jeg har tekst med metadata; hvordan forstå tekst som sprog?

udpipe --> annoterer tekst som tale/setningsled/etc

lemmatization

vowpal wabbit --> noget med hashing, der giver en numerisk betegnelse for bestemte ord

manuelt, liste af ord pr dokument, stopwords (tidytext)

metode: lda topic modeling, der ser på en gruppe dokumenter, og fors

ser på det øverste (og lidt af det midterste) af tre niveauer i teksanalayse

indhold — proces — signaler 

10.1146/annurev-soc-081715-074206

sentiment analysis — for at få øje på de nedre niveauer
→ hvad synes de om det?

\subsection{Analysestrategi}

topic modeling giver hvad-de-taler-om

sentiment analysis giver hvordan-de-taler-om

tf-idf giver grad af emne-enighed? diskurs-enighed?

Men hvad betyder det?

makro eller mikro?

Bourdieu --> kapitaler
Bernstein --> koder
(lidt fra en side) positioning theory --> hvordan omtaler man andre for at positionere sig selv (og de andre)?
