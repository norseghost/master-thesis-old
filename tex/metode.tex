\part{Metodiske og analytiske overvejelser}\label{part:method}

\epigraph{\itshape
“Når jeg anvender et ord,” sagde Klumpe-Dumpe temmelig hånligt, “så betyder det lige netop, hvad jeg vil have, det skal betyde - hverken mere eller mindre.”

“Men spørgsmålet er,” sagde Alice, “om du kan få ordene til at betyde vidt forskellige ting.”

“Spørgsmålet er,” sagde Klumpe-Dumpe, “hvem det er, der bestemmer - det er det, der er det afgørende.”
}{Lewis Carroll, \textit{Bag spejlet, og hvad Alice fandt der} (\citeyear{carrollAliceEventyrlandOg1977})}

Ord er bærende for meget af den mellemenneskelige kommunikation.
Nu kan de færreste af os tillade os en så egenrådig tilgang til ords betydning som Klumpe-Dumpe.
Ord har kun den mening, vi I fællesskab lægger i dem.
Denne mening er ikke nødvendigvis statisk, og nye fænomener nødvendiggør også til tider nye begreber.


Et sådant forholdsvist nyt fænomen er computerbehandling af tekst.
Dette går gerne under betegnelser som “Natural Language Processing” — eller NLP\footnote{Ej at forveksle med den anden NLP; “NeuroLinguistic Programming”}, eller tekstmining\footnote{Der er semantiske forskelle i disse samlebegreber; men de er ikke gensidigt udelukkende}.
Muligheden for, at lade computerprogrammer og algoritmer tygge sig igennem store mængder sprog har ikke kun ført til, at Google og Facebook ved mere om dig en din familie.
Den har også banet vej for, en ny måde at gøre sociologi på — den “algoritmiske sociologi” (eng. “computational sociology”)\todo{kilde: computational sociology}.
Dette speciale lægger sig i denne voksende tradition.

Men computeroptællinger alene gør ikke en sociologisk undersøgelse.
Der er stadigvæk behov for begrebsrammer, der kan underbygge en forståelse af de samspilsprocesser der udgør kommunikation.
Jeg vil i denne opgave benytte mig af positioneringsteori; der ser på talehandlinger i en socialt konstrueret virkelighed\todo{kilde: pos theory}.

\todo{knyt Klumpe-Dumpe og at det gælder, at bestemme, op til pos theory}

Ved at se på tekst som repræsentationer af social samhandling, kan man udlede meget af det sociale spil der har foregået.
bundet op på tekstens niveau af nærhed til det aktuelle sociale spil; some er meget umiddelbar, og kan udgøre en helhed i sig selv; en selvbiografi vil risikere post-hoc selvredaktion det aktuelle sociale spil; some er meget umiddelbar, og kan udgøre en helhed i sig selv; en selvbiografi vil risikere post-hoc selvredaktion\todo{omskriv tekst ift socialt spil}
For dette speciales vedkommende, hvor teksten er referater af folketingstaler, er afstanden til det spil, der foregår på den politiske arena forholdsvis kort.
Derved ser jeg gode muligheder for, at kunne danne insigter i det politiske spil i min analyse.



 \todo{en bedre overgang}Jeg vi i det følgende beskrive mit datagrundlag, min behandling af dette; og hvilke betydninger dette har for min analyse.
Derefter vil jeg præsentere positioneringsteori, og hvorfor jeg mener, den giver en anvendelig begrebsamme til at drage indsigter ud af min analyse.

\section{Dataindsamling og analyse}\label{sec:data}

Jeg vil, som nævnt, foretage en gennemgang af folketingstaler fra 1953 indtil primo 2020.
Datasættet er dannet ved, at hente mødereferater i PDF-form fra folketingstidende.dk, og er derefter blevet omdannet til tekstformat, og forsynet med metadata i form af fx mødets titel, dato, talerens navm og partiforhold  \autocite{pedersenFolketinget2019}.

En sådan indsamlingsproces er ikke uden fejlkilder. Konvertering fra PDF til tekstformat kan medføre fejl i overgangen, og uregelmæssigheder i dokumentet kan fx medføre mis-attributering af talere.\todo{kilde: PDF ingest fejlkilder}
Dette for at understrege, at en analyse forholder sig til, de data der foreligger.
En gennemgang af tilfældige taler viser nogle tydelige tekstgenkendelsesfejl, for eksempel.
Denne kilde til usikkerhed er det vigtigt at være bevidst.

\subsection{Den algoritmiske sociologis muligheder og begrensninger}

Tekst er kun et subset af kommunikation; der er dele af det sociale spil der kun foregår IRL, og tekst er en form for abstraheret repræsentation af disse \autocite[s. 22]{evansMachineTranslationMining2016}

begrensninger først:

Computeren har ikke samme sociale og historiske kontekst som fx i \citeauthor{juulDiskurserOmUngdom2013} i \citetitle{juulDiskurserOmUngdom2013}; og kan give en mere overfladisk analyse.

muligheder:

strukturerede klassifikationsalgoritmer kan give pålidelige klassifikationer af massive samlinger af tekst, der langt overgår den enkelte forskers kapacitet.
Ustrukturerede algoritmer kan afdække strukturer i tekstmaterialet, der kan have sociologisk interesse.
De seneste års udviklinger indenfor analyseteknologier kan også udlede mening af fx sætningsstrukturer og kontekst, til en vis grad \autocite[s. 22]{evansMachineTranslationMining2016}.

NLP - behandling af naturligt sprog;  ikke nødvendigvis kun tekst. udleder semantisk betydning af sprog - grammatik, sœtningsstruktur, osv. Dette er dog krævende beregninger!
tekst mining - udledning af data fra en samling tekst. Ord rystes sammen, uden hensyntagen til deres placering i dokumentet - typisk kaldet en Bag of Words-model. Dog kan nogle modeller for tekstanalyse se på n-grams (n ord der forekommer sammen), eller skip-grams (ord der er n ord fra hinanden) som analyseobjekt. \todo{skal jeg bruge n/skip grams mon?}
I tekstmining laver man matematiske modeller over forhold mellem ord og dokumenter.
Sådan en analyse kan drage nytte af NLP, idet det semantisk rigt data er nemmere at udlede indsigter fra.

\citeauthor{evansMachineTranslationMining2016} beskriver tre niveauer indenfor tekstanalyse.
\begin{itemize}
  \item
    Man kan se på tekstens indhold, overordnet set, og udlede kollektive tankemønstre og beslutningsprocesser
  \item
    man kan se på kommunikationsprocessen
  \item
    man kan se på sociale signaler


vowpal wabbit --> noget med hashing, der giver en numerisk betegnelse for bestemte ord
metode: lda topic modeling, der ser på en gruppe dokumenter, og fors

ser på det øverste (og lidt af det midterste) af tre niveauer i tekstanalyse \autocite{evansMachineTranslationMining2016}

\subsection{Forhåndsbearbejdning af data}\label{sec:preproc}

jeg har tekst (med metadata); hvordan kunne analysere den?

generel proces fra \autocite{kwartlerTextMiningPractice2017}

manuelt, liste af ord pr dokument, stopwords (tidytext)

lemmatization og stopwords.
Domænespecifikke stopwords pákrævede - mange ministre fx

\section{Analysestrategi}

topic modeling giver hvad-de-taler-om

sentiment analysis giver hvordan-de-taler-om

tf-idf giver grad af emne-enighed? diskurs-enighed?

Men hvad betyder det?

makro eller mikro?

Bourdieu --> kapitaler
Bernstein --> koder
positioning theory h-> hvordan omtaler man andre for at positionere sig selv (og de andre)?
