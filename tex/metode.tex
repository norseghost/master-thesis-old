\section{Metode}

\subsection{Dataindsamling og analyse}

Jeg vil foretage en gennemgang af folketingstaler fra 1953 indtil nu.

Data er hentet via scraping af folketingstidende.dk \autocite{pedersenFolketinget2019}

fejlkilder: ocr/tekstkonverteringsfejl; mis-attributering;

databearbejdning - jeg har tekst med metadata; hvordan forstå tekst som sprog?

generel proces fra \autocite{kwartlerTextMiningPractice2017}



udpipe -->  annoterer tekst som tale/setningsled/etc

vs Bag-of-words analyse

lemmatization og stopwords.
Domænespecifikke stopwords pákrævede - mange ministre fx

vowpal wabbit --> noget med hashing, der giver en numerisk betegnelse for bestemte ord

manuelt, liste af ord pr dokument, stopwords (tidytext)

metode: lda topic modeling, der ser på en gruppe dokumenter, og fors

ser på det øverste (og lidt af det midterste) af tre niveauer i tekstanalyse \autocite{evansMachineTranslationMining2016}

indhold — proces — signaler 

BoW analyse ser på indhold

sentiment analysis — for at få øje på de nedre niveauer
→ hvad synes de om det? (udpipe forsøger)

\subsection{Analysestrategi}

topic modeling giver hvad-de-taler-om

sentiment analysis giver hvordan-de-taler-om

tf-idf giver grad af emne-enighed? diskurs-enighed?

Men hvad betyder det?

makro eller mikro?

Bourdieu --> kapitaler
Bernstein --> koder
positioning theory --> hvordan omtaler man andre for at positionere sig selv (og de andre)?
